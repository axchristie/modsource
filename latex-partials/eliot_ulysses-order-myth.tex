\title{\emph{Ulysses}, Order, and Myth}
\author{}
\date{1923}

% This line adds an entry for each work into the table of contents.
%\addcontentsline{toc}{chapter}{\emph{Ulysses}, Order, and Myth,  (1923)}
\addcontentsline{toc}{chapter}{\emph{Ulysses}, Order, and Myth (1923) \newline T. S. Eliot }

\renewcommand{\chaptername}{Eliot, \emph{Ulysses}, Order, and Myth}

\thispagestyle{plain}

%\chapter[\emph{Ulysses}, Order, and Myth (1923) \newline T. S. Eliot]{\emph{Ulysses}, Order, and Myth}

% BEGIN KLUDGY TITLE BIT %%%%%%%%%%%%%%%%%%%%%%%%%%%%%%%%%%%%%%%%%%%%%%%%%%%%%%%
% We're not using \maketitle; instead this is Chris's own 
% kludgey way of outputting the title. Note the 
% uncomfortable amount of finagling with \linespread, \noindent
% and \vspace to make it look okay.
\begin{raggedright}
{\Large \linespread{1.0} \noindent \textbf{\emph{Ulysses}, Order, and Myth} \par} 

{\large T. S. Eliot \par} 

\vspace{0.5em}
\end{raggedright}

\begin{raggedleft}
{\large \linespread{1.2} (1923) \par}
\end{raggedleft}
\vspace{1em}
% END KLUDGY TITLE BIT %%%%%%%%%%%%%%%%%%%%%%%%%%%%%%%%%%%%%%%%%%%%%%%%%%%%%%%%%

%\maketitle

%\section{ \emph{Ulysses}, Order, and Myth    }






Mr.~Joyce's book has been out long enough for no more general expression
of praise, or expostulation with its detractors, to be necessary; and it
has not been out long enough for any attempt at a complete measurement
of its place and significance to be possible. All that one can usefully
do at this time, and it is a great deal to do, for such a book, is to
elucidate any aspect of the book---and the number of aspects is
indefinite---which has not yet been fixed. I hold this book to be the
most important expression which the present age has found; it is a book
to which we are all indebted, and from which none of us can escape.
These are postulates for anything that I have to say about it, and I
have no wish to waste the reader's time by elaborating my eulogies; it
has given me all the surprise, delight, and terror that I can require,
and I will leave it at that.

Among all the criticisms I have seen of the book, I have seen
nothing---unless we except, in its way, M Valery Larbaud's valuable
paper which is rather an Introduction than a criticism---which seemed to
me to appreciate the significance of the method employed---the parallel
to the Odyssey, and the use of appropriate styles and symbols to each
division. Yet one might expect this to be the first peculiarity to
attract attention; but it has been treated as an amusing dodge, or
scaffolding erected by the author for the purpose of disposing his
realistic tale, of no interest in the completed structure. The criticism
which Mr.~Aldington directed upon \emph{Ulysses} several years ago seems
to me to fail by this oversight---but, as Mr Aldington wrote before the
complete work had appeared, fails more honourably than the attempts of
those who had the whole book before them. Mr.~Aldington treated
Mr.~Joyce as a prophet of chaos; and wailed at the flood of Dadaism
which his prescient eye saw bursting forth at the tap of the magician's
rod. Of course, the influence which Mr.~Joyce's book may have is from my
point of view an irrelevance. A very great book may have a very bad
influence indeed; and a mediocre book may be in the event most salutary.
The next generation is responsible for its own soul; a man of genius is
responsible to his peers, not to a studio-full of uneducated and
undisciplined coxcombs. Still, Mr.~Aldington'ss pathetic solicitude for
the haif-witted seems to me to carry certain implications about the
nature of the book itself to which I cannot assent; and this is the
important issue. He finds the book, if I understand him, to be an
invitation to chaos, and an expression of feelings which are perverse,
partial, and a distortion of reality. But unless I quote Mr.~Aldington's
words I am likely to falsify. ``I say, moreover,'' he says,\footnote{English
  Review, April 1921.} ``that when Mr.~Joyce, with his marvellous gifts,
uses them to disgust us with mankind, he is doing something which is
false and a libel on humanity.'' It is somewhat similar to the opinion
of the urbane Thackeray upon Swift. ``As for the moral, I think it
horrible, shameful, unmanly, blasphemous: and giant and great as this
Dean is, I say we should hoot him.'' (This, of the conclusion of the
Voyage to the Houyhnhnms---which seems to me one of the greatest
triumphs that the human soul has ever achieved.---It is true that
Thackeray later pays Swift one of the finest tributes that a man has
ever given or received: ``So great a man he seems to me that thinking of
him is like thinking of an empire falling.'' And Mr.~Aldington, in his
time, is almost equally generous.)

Whether it is possible to libel humanity (in distinction to libel in the
usual sense, which is libelling an individual or a group in contrast
with the rest of humanity) is a question for philosophical societies to
discuss; but of course if \emph{Ulysses} were a ``libel'' it would
simply be a forged document, a powerless fraud, which would never have
extracted from Mr.~Aldington a moment's attention. I do not wish to
linger over this point: the interesting question is that begged by
Mr.~Aldington when he refers to Mr.~Joyce's ``great \emph{undisciplined}
talent.''

I think that Mr.~Aldington and I are more or less agreed as to what we
want in principle, and agreed to call it classicism. It is because of
this agreement that I have chosen Mr.~Aldington to attack on the present
issue. We are agreed as to what we want, but not as to how to get it, or
as to what contemporary writing exhibits a tendency in that direction.
We agree, I hope, that ``classicism'' is not an alternative to
``romanticism,'' as of political parties, Conservative and Liberal,
Republican and Democrat, on a ``turn-the-rascals-out'' platform. It is a
goal toward which all good literature strives, so far as it is good,
according to the possibilities of its place and time. One can be
``classical,'' in a sense, by turning away from nine-tenths of the
material which lies at hand, and selecting only mummified stuff from a
museum---like some contemporary writers, about whom one could say some
nasty things in this connexion, if it were worth while (Mr.~Aldington is
not one of them). Or one can be classical in tendency by doing the best
one can with the material at hand. The confusion springs from the fact
that the term is applied to literature and to the whole complex of
interests and modes of behaviour and society of which literature is a
part; and it has not the same bearing in both applications. It is much
easier to be a classicist in literary criticism than in creative
art---because in criticism you are responsible only for what you want,
and in creation you are responsible for what you can do with material
which you must simply accept. And in this material I include the
emotions and feelings of the writer himself, which, for that writer, are
simply material which he must accept---not virtues to be enlarged or
vices to be diminished. The question, then, about Mr.~Joyce, is: how
much living material does he deal with, and how does he deal with it:
deal with, not as a legislator or exhorter, but as an artist?

It is here that Mr.~Joyce's parallel use of the Odyssey has a great
importance. It has the importance of a scientific discovery. No one else
has built a novel upon such a foundation before: it has never before
been necessary. I am not begging the question in calling \emph{Ulysses}
a ``novel''; and if you call it an epic it will not matter. If it is not
a novel, that is simply because the novel is a form which will no longer
serve; it is because the novel, instead of being a form, was simply the
expression of an age which had not sufficiently lost all form to feel
the need of something stricter. Mr.~Joyce has written one novel---the
\emph{Portrait}; Mr.~Wyndham Lewis has written one novel---\emph{Tarr}.
I do not suppose that either of them will ever write another ``novel.''
The novel ended with Flaubert and with James. It is, I think, because
Mr.~Joyce and Mr.~Lewis, being ``in advance'' of their time, felt a
conscious or probably unconscious dissatisfaction with the form, that
their novels are more formless than those of a dozen clever writers who
are unaware of its obsolescence.

In using the myth, in manipulating a continuous parallel between
contemporaneity and antiquity, Mr.~Joyce is pursuing a method which
others must pursue after him. They will not be imitators, any more than
the scientist who uses the discoveries of an Einstein in pursuing his
own, independent, further investigations. It is simply a way of
controlling, of ordering, of giving a shape and a significance to the
immense panorama of futility and anarchy which is contemporary history.
It is a method already adumbrated by Mr.~Yeats, and of the need for
which I believe Mr.~Yeats to have been the first contemporary to be
conscious. It is a method for which the horoscope is auspicious.
Psychology (such as it is, and whether our reaction to it be comic or
serious) ethnology, and The Golden Bough have concurred to make possible
what was impossible even a few years ago. Instead of narrative method,
we may now use the mythical method. It is, I seriously believe, a step
toward making the modern world possible for art, toward that order and
form which Mr.~Aldington so earnestly desires. And only those who have
won their own discipline in secret and without aid, in a world which
offers very little assistance to that end, can be of any use in
furthering this advance.

%\newpage