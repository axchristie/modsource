\title{A Retrospect}
\author{}
\date{1918}

% This line adds an entry for each work into the table of contents.
%\addcontentsline{toc}{chapter}{A Retrospect,  (1918)}
\addcontentsline{toc}{chapter}{A Retrospect (1918) \newline Ezra Pound }

\renewcommand{\chaptername}{Pound, A Retrospect}

\thispagestyle{plain}

%\chapter[A Retrospect (1918) \newline Ezra Pound]{A Retrospect}

% BEGIN KLUDGY TITLE BIT %%%%%%%%%%%%%%%%%%%%%%%%%%%%%%%%%%%%%%%%%%%%%%%%%%%%%%%
% We're not using \maketitle; instead this is Chris's own 
% kludgey way of outputting the title. Note the 
% uncomfortable amount of finagling with \linespread, \noindent
% and \vspace to make it look okay.
\begin{raggedright}
{\Large \linespread{1.0} \noindent \textbf{A Retrospect} \par} 

{\large Ezra Pound \par} 

\vspace{0.5em}
\end{raggedright}

\begin{raggedleft}
{\large \linespread{1.2} (1918) \par}
\end{raggedleft}
\vspace{1em}
% END KLUDGY TITLE BIT %%%%%%%%%%%%%%%%%%%%%%%%%%%%%%%%%%%%%%%%%%%%%%%%%%%%%%%%%

%\maketitle

%\section{ A Retrospect    }






There has been so much scribbling about a new fashion in poetry, that I
may perhaps be pardoned this brief recapitulation and retrospect.

In the spring or early summer of 1912, ``H.D.,'' Richard Aldington and
myself decided that we were agreed upon the three principles following:

\begin{enumerate}
\def\labelenumi{\arabic{enumi}.}
\itemsep1pt\parskip0pt\parsep0pt
\item
  Direct treatment of the ``thing'' whether subjective or objective.
\item
  To use absolutely no word that does not contribute to the
  presentation.
\item
  As regarding rhythm: to compose in the sequence of the musical phrase,
  not in sequence of a metronome.
\end{enumerate}

Upon many points of taste and of predilection we differed, but agreeing
upon these three positions we thought we had as much right to a group
name, at least as much right, as a number of French ``schools''
proclaimed by Mr.~Flint in the August number of Harold Monro's magazine
for 1911.

This school has since been ``joined'' or ``followed'' by numerous people
who, whatever their merits, do not show any signs of agreeing with the
second specification. Indeed vers libre has become as prolix and as
verbose as any of the flaccid varieties that preceded it. It has brought
faults of its own. The actual language and phrasing is often as bad as
that of our elders without even the excuse that the words are shovelled
in to fill a metric pattern or to complete the noise of a rhyme-sound.
Whether or no the phrases followed by the followers are musical must be
left to the reader's decision. At times I can find a marked metre in
``vers libres,'' as stale and hackneyed as any pseudo-Swinburnian, at
times the writers seem to follow no musical structure whatever. But it
is, on the whole, good that the field should be ploughed. Perhaps a few
good poems have come from the new method, and if so it is justified.

\subsection{. . .}\label{section}

Criticism is not a circumscription or a set of prohibitions. It provides
fixed points of departure. It may startle a dull reader into alertness.
That little of it which is good is mostly in stray phrases; or if it be
an older artist helping a younger it is in great measure but rules of
thumb, cautions gained by experience.

I set together a few phrases on practical working about the time the
first remarks on imagisme were published. The first use of the word
``Imagiste'' was in my note to T. E. Hulme's five poems, printed at the
end of my ``Ripostes'' in the autumn of 1912. I reprint my cautions from
\emph{Poetry} for March, 1913.

\subsubsection{A FEW DON'TS}\label{a-few-donts}

An ``Image'' is that which presents an intellectual and emotional
complex in an instant of time. I use the term ``complex'' rather
technical sense employed by the newer psychologists, such as Hart,
though we might not agree absolutely in our application.

It is the presentation of such a ``complex'' instantaneously which gives
that sense of sudden liberation; that sense of freedom from time limits
and space limits; that sense of sudden growth, which we experience in
the presence of the greatest works of art.

It is better to present one Image in a lifetime than to produce
voluminous works.

All this, however, some may consider open to debate. The immediate
necessity is to tabulate A LIST OF DON'TS for those beginning to write
verses. I can not put all of them into Mosaic negative.

To begin with, consider the three propositions (demanding direct
treatment, economy of words, and the sequence of the musical phrase),
not as dogma---never consider anything as dogma---but as the result of
long contemplation, which, even if it is some one else's contemplation,
may be worth consideration.

Pay no attention to the criticism of men who have never themselves
written a notable work. Consider the discrepancies between the actual
writing of the Greek poets and dramatists, and the theories of the
Graeco-Roman grammarians, concocted to explain their metres.

\subsubsection{LANGUAGE}\label{language}

Use no superfluous word, no adjective, which does not reveal something.

Don't use such an expression as ``dim lands of \emph{peace}.'' It dulls
the image. It mixes an abstraction with the concrete. It comes from the
writer's not realizing that the natural object is always the
\emph{adequate} symbol.

Go in fear of abstractions. Do not retell in mediocre verse what has
already been done in good prose. Don't think any intelligent person is
going to be deceived when you try to shirk all the difficulties of the
unspeakably difficult art of good prose by chopping your composition
into line lengths.

What the expert is tired of today the public will be tired of tomorrow.

Don't imagine that the art of poetry is any simpler than the art of
music, or that you can please the expert before you have spent at least
as much effort on the art of verse as an average piano teacher spends on
the art of music.

Be influenced by as many great artists as you can, but have the decency
either to acknowledge the debt outright, or to try to conceal it.

Don't allow ``influence'' to mean merely that you mop up the particular
decorative vocabulary of some one or two poets whom you happen to
admire. A Turkish war correspondent was recently caught red-handed
babbling in his dispatches of ``dove-grey'' hills, or else it was
``pearl-pale,'' I can not remember.

Use either no ornament or good ornament.

\subsubsection{RHYTHM AND RHYME}\label{rhythm-and-rhyme}

Let the candidate fill his mind with the finest cadences he can
discover, preferably in a foreign language\footnote{This is for rhythm,
  his vocabulary must of course be found in his native tongue.} so that
the meaning of the words may be less likely to divert his attention from
the movement; e.g.~Saxon charms, Hebridean Folk Songs, the verse of
Dante, and the lyrics of Shakespeare---if he can dissociate the
vocabulary from the cadence. Let him dissect the lyrics of Goethe coldly
into their component sound values, syllables long and short, stressed
and unstressed, into vowels and consonants.

It is not necessary that a poem should rely on its music, but if it does
rely on its music that music must be such as will delight the expert.

Let the neophyte know assonance and alliteration, rhyme immediate and
delayed, simple and polyphonic, as a musician would expect to know
harmony and counterpoint and all the minutiae of his craft. No time is
too great to give to these matters or to any one of them, even if the
artist seldom have need of them.

Don't imagine that a thing will ``go'' in verse just because it's too
dull to go in prose.

Don't be ``viewy''---leave that to the writers of pretty little
philosophic essays. Don't be descriptive; remember that the painter can
describe a landscape much better than you can, and that he has to know a
deal more about it.

When Shakespeare talks of the ``Dawn in russet mantle clad'' he presents
something which the painter does not present. There is in this line of
his nothing that one can call description; he presents.

Consider the way of the scientists rather than the way of an advertising
agent for a new soap.

The scientist does not expect to be acclaimed as a great scientist until
he has \emph{discovered} something. He begins by learning what has been
discovered already. He goes from that point onward. He does not bank on
being a charming fellow personally. He does not expect his friends to
applaud the results of his freshman class work. Freshmen in poetry are
unfortunately not confined to a definite and recognizable class room.
They are ``all over the shop.'' Is it any wonder ``the public is
indifferent to poetry?''

Don't chop your stuff into separate \emph{iambs}. Don't make each line
stop dead at the end, and then begin every next line with a heave. Let
the beginning of the next line catch the rise of the rhythm wave, unless
you want a definite longish pause.

In short, behave as a musician, a good musician, when dealing with that
phase of your art which has exact parallels in music. The same laws
govern, and you are bound by no others.

Naturally, your rhythmic structure should not destroy the shape of your
words, or their natural sound, or their meaning. It is improbable that,
at the start, you will be able to get a rhythm-structure strong enough
to affect them very much, though you may fall a victim to all sorts of
false stopping due to line ends, and cæsurae.

The musician can rely on pitch and the volume of the orchestra. You can
not. The term harmony is misapplied in poetry; it refers to simultaneous
sounds of different pitch. There is, however, in the best verse a sort
of residue of sound which remains in the ear of the hearer and acts more
or less as an organ-base.

A rhyme must have in it some slight element of surprise if it is to give
pleasure; it need not be bizarre or curious, but it must be well used if
used at all.

Vide further Vildrac and Duhamel's notes on rhyme in ``Technique
Poetique.''

That part of your poetry which strikes upon the imaginative \emph{eye}
of the reader will lose nothing by translation into a foreign tongue;
that which appeals to the ear can reach only those who take it in the
original.

Consider the definiteness of Dante's presentation, as compared with
Milton's rhetoric. Read as much of Wordsworth as does not seem too
unutterably dull.\footnote{Vide infra.}

If you want the gist of the matter go to Sappho, Catullus, Villon, Heine
when he is in the vein, Gautier when he is not too frigid; or, if you
have not the tongues, seek out the leisurely Chaucer. Good prose will do
you no harm, and there is good discipline to be had by trying to write
it.

Translation is likewise good training, if you find that your original
matter ``wobbles'' when you try to rewrite it. The meaning of the poem
to be translated can not ``wobble.''

If you are using a symmetrical form, don't put in what you want to say
and then fill up the remaining vacuums with slush.

Don't mess up the perception of one sense by trying to define it in
terms of another. This is usually only the result of being too lazy to
find the exact word. To this clause there are possibly exceptions.

The first three simple prescriptions will throw out nine-tenths of all
the bad poetry now accepted as standard and classic; and will prevent
you from many a crime of production.

``\ldots{} \emph{Mais d'abord il faut être un poète},'' as MM. Duhamel
and Vildrac have said at the end of their little book, ``Notes sur la
Technique Poetique.''

\subsection{. . .}\label{section-1}

Since March, 1913, Ford Madox Hueffer has pointed out that Wordsworth
was so intent on the ordinary or plain word that he never thought of
hunting for \emph{le mot juste}.

John Butler Yeats has handled or man-handled Wordsworth and the
Victorians, and his criticism, contained in letters to his son, is now
printed and available.

I do not like writing \emph{about} art, my first, at least I think it
was my first essay on the subject, was a protest against it.

\subsubsection[PROLEGOMENA]{PROLEGOMENA\footnote{\emph{Poetry and Drama}
  (then the \emph{Poetry Review}, edited by Harold Monro), Feb., 1912.}}\label{prolegomena}

Time was when the poet lay in a green field with his head against a tree
and played his diversion on a ha'penny whistle, and Cæsar's predecessors
conquered the earth, and the predecessors of golden Crassus embezzled,
and fashions had their say, and let him alone. And presumably he was
fairly content in this circumstance, for I have small doubt that the
occasional passerby, being attracted by curiosity to know why any one
should lie under a tree and blow diversion on a ha'penny whistle, came
and conversed with him, and that among these passers-by there was on
occasion a person of charm or a young lady who had not read ``Man and
Superman''; and looking back upon this naïve state of affairs we call it
the age of gold.

Metastasio, and he should know if any one, assures us that this age
endures---even though the modern poet is expected to holloa his verses
down a speaking tube to the editors of cheap magazines---S.S. McClure,
or some one of that sort---even though hordes of authors meet in
dreariness and drink healths to the ``Copyright Bill''; even though
these things be, the age of gold pertains. Imperceivably, if you like,
but pertains. You meet unkempt Amyclas in a Soho restaurant and chant
together of dead and forgotten things---it is a manner of speech among
poets to chant of dead, half-forgotten things, there seems no special
harm in it; it has always been done---and it's rather better to be a
clerk in the Post Office than to look after a lot of stinking, verminous
sheep---and at another hour of the day one substitutes the drawing-room
for the restaurant and tea is probably more palatable than mead and
mare's milk, and little cakes than honey. And in this fashion one
survives the resignation of Mr.~Balfour, and the iniquities of the
American customs-house, \emph{e quel bufera infernal}, the periodical
press. And then in the middle of it, there being apparently no other
person at once capable and available one is stopped and asked to explain
oneself.

I begin on the chord thus querulous, for I would much rather lie on what
is left of Catullus' parlour floor and speculate the azure beneath it
and the hills off to Salo and Riva with their forgotten gods moving
unhindered amongst them, than discuss any processes and theories of art
whatsoever. I would rather play tennis. I shall not argue.

\subsubsection{CREDO}\label{credo}

\emph{Rhythm}.---I believe in an ``absolute rhythm,'' a rhythm, that is,
in poetry which corresponds exactly to the emotion or shade of emotion
to be expressed. A man's rhythm must be interpretative, it will be,
therefore, in the end, his own, uncounterfeiting, uncounterfeitable.

\emph{Symbols}.---I believe that the proper and perfect symbol is the
natural object, that if a man use ``symbols'' he must so use them that
their symbolic function does not obtrude; so that \emph{a} sense, and
the poetic quality of the passage, is not lost to those who do not
understand the symbol as such, to whom, for instance, a hawk is a hawk.

\emph{Technique}.---I believe in technique as the test of a man's
sincerity; in law when it is ascertainable; in the trampling down of
every convention that impedes or obscures the determination of the law,
or the precise rendering of the impulse.

\emph{Form}.---I think there is a ``fluid'' as well as a ``solid''
content, that some poems may have form as a tree has form, some as water
poured into a vase. That most symmetrical forms have certain uses. That
a vast number of subjects cannot be precisely, and therefore not
properly rendered in symmetrical forms.

``Thinking that alone worthy wherein the whole art is
employed,''\footnote{Dante, De Volgari Eloquio.} I think the artist
should master all known forms and systems of metric, and I have with
some persistence set about doing this, searching particularly into those
periods wherein the systems came to birth or attained their maturity. It
has been complained, with some justice, that I dump my note-books on the
public. I think that only after a long struggle will poetry attain such
a degree of development, of, if you will, modernity, that it will
vitally concern people who are accustomed, in prose, to Henry James and
Anatole France, in music to Debussy. I am constantly contending that it
took two centuries of Provençe and one of Tuscany to develop the media
of Dante's masterwork, that it took the latinists of the Renaissance,
and the Pleiade, and his own age of painted speech to prepare
Shakespeare his tools. It is tremendously important that great poetry be
written, it makes no jot of difference who writes it. The experimental
demonstrations of one man may save the time of many---hence my furore
over Arnaut Daniel---if a man's experiments try out one new rime, or
dispense conclusively with one iota of currently accepted nonsense, he
is merely playing fair with his colleagues when he chalks up his result.

No man ever writes very much poetry that ``matters.'' In bulk, that is,
no one produces much that is final, and when a man is not doing this
highest thing, this saying the thing once for all and perfectly; when he
is not matching Ποικιλόθρον', ὰθάνατ' Ἀφρόδιτα, or ``Hist---said Kate
the Queen,'' he had much better be making the sorts of experiment which
may be of use to him in his later work, or to his successors.

``The lyf so short, the craft so long to lerne.'' It is a foolish thing
for a man to begin his work on a too narrow foundation, it is a
disgraceful thing for a man's work not to show steady growth and
increasing fineness from first to last.

As for ``adaptations''; one finds that all the old masters of painting
recommend to their pupils that they begin by copying masterwork, and
proceed to their own composition.

As for ``Every man his own poet.'' The more every man knows about poetry
the better. I believe in every one writing poetry who wants to; most do.
I believe in every man knowing enough of music to play ``God bless our
home'' on the harmonicum, but I do not believe in every man giving
concerts and printing his sin.

The mastery of any art is the work of a lifetime. I should not
discriminate between the ``amateur'' and the ``professional,'' or rather
I should discriminate quite often in favour of the amateur, but I should
discriminate between the amateur and the expert. It is certain that the
present chaos will endure until the Art of poetry has been preached down
the amateur gullet, until there is such a general understanding of the
fact that poetry is an art and not a pastime; such a knowledge of
technique; of technique of surface and technique of content, that the
amateurs will cease to try to drown out the masters.

If a certain thing was said once for all in Atlantis or Arcadia, in 450
Before Christ or in 1290 after, it is not for us moderns to go saying it
over, or to go obscuring the memory of the dead by saying the same thing
with less skill and less conviction.

My pawing over the ancients and semi-ancients has been one struggle to
find out what has been done, once for all, better than it can ever be
done again, and to find out what remains for us to do, and plenty does
remain, for if we still feel the same emotions as those which launched
the thousand ships, it is quite certain that we come on these feelings
differently, through different nuances, by different intellectual
gradations. Each age has its own abounding gifts, yet only some ages
transmute them into matter of duration. No good poetry is ever written
in a manner twenty years old, for to write in such a manner shows
conclusively that the writer thinks from books, convention and
\emph{cliché}, and not from life, yet a man feeling the divorce of life
and his art may naturally try to resurrect a forgotten mode if he find
in that mode some leaven, or if he think he sees in it some element
lacking in contemporary art which might unite that art again to its
sustenance, life.

In the art of Daniel and Cavalcanti, I have seen that precision which I
miss in the Victorians---that explicit rendering, be it of external
nature, or of emotion. Their testimony is of the eyewitness, their
symptoms are first hand.

As for the nineteenth century, with all respect to its achievements, I
think we shall look back upon it as a rather blurry, messy sort of a
period, a rather sentimentalistic, mannerish sort of a period. I say
this without any self-righteousness, with no self-satisfaction.

As for there being a ``movement'' or my being of it, the conception of
poetry as a ``pure art'' in the sense in which I use the term, revived
with Swinburne. From the puritanical revolt to Swinburne, poetry had
been merely the vehicle---yes, definitely, Arthur Symons' scruples and
feelings about the word not withholding--the ox-cart and post-chaise for
transmitting thoughts poetic or otherwise. And perhaps the ``great
Victorians,'' though it is doubtful, and assuredly the ``nineties''
continued the development of the art, confining their improvements,
however, chiefly to sound and to refinements of manner.

Mr.~Yeats has once and for all stripped English poetry of its
perdamnable rhetoric. He has boiled away all that is not poetic---and a
good deal that is. He has become a classic in his own lifetime and
\emph{nel mezzo del cammin}. He has made our poetic idiom a thing
pliable, a speech without inversions.

Robert Bridges, Maurice Hewlett and Frederic Manning are\footnote{(Dec.,
  1911.)} in their different ways seriously concerned with overhauling
the metric, in testing the language and its adaptability to certain
modes. Ford Hueffer is making some sort of experiments in modernity. The
Provost of Oriel continues his translation of the \emph{Divina
Commedia}.

As to Twentieth century poetry, and the poetry which I expect to see
written during the next decade or so, it will, I think, move against
poppy-cock, it will be harder and saner, it will be what Mr.~Hewlett
calls ``nearer the bone.'' It will be as much like granite as it can be,
its force will lie in its truth, its interpretative power (of course,
poetic force does always rest there); I mean it will not try to seem
forcible by rhetorical din, and luxurious riot. We will have fewer
painted adjectives impeding the shock and stroke of it. At least for
myself, I want it so, austere, direct, free from emotional slither.

\subsection{. . .}\label{section-2}

What is there now, in 1917, to be added?

\subsubsection{RE VERS LIBRE}\label{re-vers-libre}

I think the desire for vers libre is due to the sense of quantity
reasserting itself after years of starvation. But I doubt if we can take
over, for English, the rules of quantity laid down for greek and latin,
mostly by latin grammarians.

I think one should write vers libre only when one ``must,'' that is to
say, only when the ``thing'' builds up a rhythm more beautiful than that
of set metres, or more real, more a part of the emotion of the
``thing,'' more germane, intimate, interpretative than the measure of
regular accentual verse; a rhythm which discontents one with set iambic
or set anapaestic.

Eliot has said the thing very well when he said, ``No \emph{vers} is
\emph{libre} for the man who wants to do a good job.''

As a matter of detail, there is vers libre with accent heavily marked as
a drum-beat (as par example my ``Dance Figure''), and on the other hand
I think I have gone as far as can profitably be gone in the other
direction (and perhaps too far). I mean I do not think one can use to
any advantage rhythms much more tenuous and imperceptible than some I
have used. I think progress lies rather in an attempt to approximate
classical quantitative metres (NOT to copy them) than in a carelessness
regarding such things.\footnote{Let me date this statement 20. Aug.,
  1917.}

. . .

I agree with John Yeats on the relation of beauty to certitude. I prefer
satire, which is due to emotion, to any sham of emotion.

I have had to write, or at least I have written a good deal about art,
sculpture, painting and poetry. I have seen what seemed to me the best
of contemporary work reviled and obstructed. Can any one write prose of
permanent or durable interest when he is merely saying for one year what
nearly every one will say at the end of three or four years? I have been
battistrada for a sculptor, a painter, a novelist, several poets. I
wrote also of certain French writers in \emph{The New Age} in nineteen
twelve or eleven.

I would much rather that people would look at Brzeska's sculpture and
Lewis' drawings, and that they would read Joyce, Jules Romains, Eliot,
than that they should read what I have said of these men, or that I
should be asked to republish argumentative essays and reviews.

All that the critic can do for the reader or audience or spectator is to
focus his gaze or audition. Rightly or wrongly I think my blasts and
essays have done their work, and that more people are now likely to go
to the sources than are likely to read this book.

Jammes' ``Existences'' in ``La Triomphe de la Vie'' is available. So are
his early poems. I think we need a convenient anthology rather than
descriptive criticism. Carl Sandburg wrote me from Chicago, ``It's hell
when poets can't afford to buy each other's books.'' Half the people who
care, only borrow. In America so few people know each other that the
difficulty lies more than half in distribution. Perhaps one should make
an anthology: Romains' ``Un Être en Marche'' and ``Prières,'' Vildrac's
``Visite.'' Retrospectively the fine wrought work of La Forgue, the
flashes of Rimbaud, the hard-bit lines of Tristan Corbière, Tailhade's
sketches in ``Poèmes Aristophanesques,'' the ``Litanies'' of DeGourmont.

\subsection{. . .}\label{section-3}

It is difficult at all times to write of the fine arts, it is almost
impossible unless one can accompany one's prose with many reproductions.
Still I would seize this chance or any chance to reaffirm my belief in
Wyndham Lewis' genius, both in his drawings and his writings. And I
would name an out of the way prose book, the ``Scenes and Portraits'' of
Frederic Manning, as well as James Joyce's short stories and novel,
``Dubliners'' and the now well known ``Portrait of the Artist,'' as well
as Lewis' ``Tarr,'' if, that is, I may treat my strange reader as if he
were a new friend come into the room, intent on ransacking my bookshelf.

\subsubsection{ONLY EMOTION ENDURES}\label{only-emotion-endures}

``Only emotion endures.'' Surely it is better for me to name over the
few beautiful poems that still ring in my head than for me to search my
flat for back numbers of periodicals and rearrange all that I have said
about friendly and hostile writers.

The first twelve lines of Padraic Colum's ``Drover''; his ``O Woman
shapely as a swan, on your account I shall not die''; Joyce's ``I hear
an army''; the lines of Yeats that ring in my head and in the heads of
all young men of my time who care for poetry: Braseal and the Fisherman,
``The fire that stirs about her when she stirs''; the later lines of
``The Scholars,'' the faces of the Magi; William Carlos Williams'
``Postlude,'' Aldington's version of ``Atthis,'' and ``H. D.'s'' waves
like pine tops, and her verse in ``Des Imagistes'' the first anthology;
Hueffer's ``How red your lips are'' in his translation from Von der
Vogelweide, his ``Three Ten,'' the general effect of his ``On Heaven'';
his sense of the prose values or prose qualities in poetry; his ability
to write poems that will sing to music, as distinct from poems that
half-chant and are spoiled by a musician's additions; beyond these a
poem by Alice Corbin, ``One City Only,'' and another ending ``But
sliding water over a stone.'' These things have worn smooth in my head
and I am not through with them, nor with Aldington's ``In Via Sestina''
nor his other poems in ``Des Imagistes'' though people have told me
their flaws. It may be that their content is too much embedded in me for
me to look back at the words.

I am almost a different person when I come to take up the argument for
Eliot's poems.

\subsection{. . .}\label{section-4}

%\newpage