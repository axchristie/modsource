\title{On the Modern Element in Literature}
\author{}
\date{1869}

% This line adds an entry for each work into the table of contents.
%\addcontentsline{toc}{chapter}{On the Modern Element in Literature,  (1869)}
\addcontentsline{toc}{chapter}{On the Modern Element in Literature (1869) \newline Matthew Arnold }

\renewcommand{\chaptername}{Arnold, On the Modern Element in Literature}

\thispagestyle{plain}

%\chapter[On the Modern Element in Literature (1869) \newline Matthew Arnold]{On the Modern Element in Literature}

% BEGIN KLUDGY TITLE BIT %%%%%%%%%%%%%%%%%%%%%%%%%%%%%%%%%%%%%%%%%%%%%%%%%%%%%%%
% We're not using \maketitle; instead this is Chris's own 
% kludgey way of outputting the title. Note the 
% uncomfortable amount of finagling with \linespread, \noindent
% and \vspace to make it look okay.
\begin{raggedright}
{\Large \linespread{1.0} \noindent \textbf{On the Modern Element in Literature} \par} 

{\large Matthew Arnold \par} 

\vspace{0.5em}
\end{raggedright}

\begin{raggedleft}
{\large \linespread{1.2} (1869) \par}
\end{raggedleft}
\vspace{1em}
% END KLUDGY TITLE BIT %%%%%%%%%%%%%%%%%%%%%%%%%%%%%%%%%%%%%%%%%%%%%%%%%%%%%%%%%

%\maketitle

%\section{ On the Modern Element in Literature    }






{[}What follows was delivered as an inaugural lecture in the Poetry
Chair at Oxford. It was never printined, but there appeared at the time
several comments on it from critics who had either heard it, or heard
reports about it. It was meant to be followed and completed by a course
of lectures developing the subject entirely, and some of these were
given. But the course was broken off because I found my knowledge
unsufficient for treating in a solid way many portions of the subject
chosen. The inaugural lecture, however, treating a portion of the
subject where my knowledge was perhaps less insufficient, and where
besides my hearers were better able to help themselves out from their
own knowledge, is here printed. No one feels the imperfection of this
sketchy and generalizing mode of treatment more than I do; and not only
is this mode of treatment less to my taste now than it was eleven years
ago, but the style too, which is that of the doctor rather than the
explorer, is a style which I have long since learnt to abandon.
Nevertheless, having written much of late about Hellenis and Hebraism,
and Hellenism being to many people almost an empty name compared with
Hebraism, I print this lecture with the hope that it may serve, in the
absence of other and fuller illustrations, to give some notion of the
Hellenic spirit and its works, and of their significant in the history
of the evolution of the human spirit in general. M. A.{]}

It is related in one of those legends which illustrate the history of
Buddhism, that a certain disciple once presented himself before his
master, Buddha, with the desire to be permitted to undertake a mission
of peculiar difficulty. The compassionate teacher represented to him the
obstacles to be surmounted and the risks to be run. Pourna---so the
disciple was called---insisted, and replied, with equal humility and
adroitness, to the successive objections of his adviser. Satisfied at
last by his answers of the fitness of his disciple, Buddha accorded to
him the desired permission; and dismissed him to his task with these
remarkable words, nearly identical with those in which he himself is
said to have been admonished by a divinity at the outset of his own
career:---``Go then, O Pourna,'' are his words; ``having been delivered,
deliver; having been consoled, console; being arrived thyself at the
farther bank, enable others to arrive there also.''

It was a moral deliverance, eminently, of which the great Oriental
reformer spoke; it was a deliverance from the pride, the sloth, the
anger, the selfishness, which impair the moralactivity of man---a
deliverance which is demanded of all individuals and in all ages. But
there is another deliverance for the human race, hardly less important,
indeed, than the first---for in the enjoyment of both united consists
man's true freedom---but demanded far less universally, and even more
rarely and imperfectly obtained; a deliverance neglected, apparently
hardly conceived, in some ages, while it has been pursued with
earnestness in others, which derive from that very pursuit their
peculiar character. This deliverance is an intellectual deliverance.

An intellectual deliverance is the peculiar demand of those ages which
are called modern; and those nations are said to be imbued with the
modern spirit most eminently in which the demand for such a deliverance
has been made with most zeal, and satisfied with most completeness. Such
a deliverance is emphatically, whether we will or no, the demand of the
age in which we ourselves live. All intellectual pursuits our age judges
according to their power of helping to satisfy this demand; of all
studies it asks, above all, the question, how far they can contribute to
this deliverance.

I propose, on this my first occasion of speaking here, to attempt such a
general survey of ancient classical literature and history as may afford
us the conviction---in presence of the doubts so often expressed of the
profitableness, in the present day, of our study of this
literature---that, even admitting to their fullest extent the legitimate
demands of our age, the literature of ancient Greece is, even for modern
times, a mighty agent of intellectual deliverance; even for modern
times, therefore, an object of indestructible interest.

But first let us ask ourselves why the demand for an intellectual
deliverance arises in such an age as the present, and in what the
deliverance itself consists? The demand arises, because our present age
has around it a copious and complex present, and behind it a copious and
complex past; it arises, because the present age exhibits to the
individual man who contemplates it the spectacle of a vast multitude of
facts awaiting and inviting his comprehension. The deliverance consists
in man's comprehension of this present and past. It begins when our mind
begins to enter into possession of the general ideas which are the law
of this vast multitude of facts. It is perfect when we have acquired
that harmonious acquiescence of mind which wefeel in contemplating a
grand spectacle that is intelligible to us; when we have lost that
impatient irritation of mind which we feel in presence of an immense,
moving, confused spectacle which, while it perpetually excites our
curiosity, perpetually baffles our comprehension.

This, then, is what distinguishes certain epochs in the history of the
human race, and our ownamongst the number;---on the one hand, the
presence of a significant spectacle to contemplate; on the other hand,
the desire to find the true point of view from which to contemplate this
spectacle. He who has found that point of view, he who adequately
comprehends this spectacle, has risen to the comprehension of his age:
he who communicates that point of view to his age, he who interprets to
it that spectacle, is one of his age's intellectual deliverers.

The spectacle, the facts, presented for the comprehension of the present
age, are indeed immense. The facts consist of the events, the
institutions, the sciences, the arts, the literatures, in which human
life has manifested itself up to the present time: the spectacle is the
collective life of humanity. And everywhere there is connexion,
everywhere there is illustration: no single event, no single literature,
is adequately comprehended except in its relation to other events, to
other literatures. The literature of ancient Greece, the literature of
the Christian Middle Age, so long as they are regarded as two isolated
literatures, two isolated growths of the human spirit, are not
adequately comprehended; and it is adequate comprehension which is the
demand of the present age. ``We must compare,''---the illustrious
Chancellor of Cambridge\footnote{The late Prince Consort.} said the
other day to his hearers at Manchester,---``we must compare the works of
other ages with those of our own age and country; that, while we feel
proud of the immense development of knowledge and power of production
which we possess, we may learn humility in contemplating the refinement
of feeling and intensity of thought manifested in the works of the older
schools.'' To know how others stand, that we may know how we ourselves
stand; and to know how we ourselves stand, that we may correct our
mistakes and achieve our deliverance---that is our problem.

But all facts, all the elements of the spectacle before us, have not an
equal value---do not merit a like attention: and it is well that they do
not, for no man would be adequate to the task of thoroughly mastering
them all. Some have more significance for us, others have less; some
merit our utmost attention in all their details, others it is sufficient
to comprehend in their general character, and then they may be
dismissed.

What facts, then, let us ask ourselves, what elements of the spectacle
before us, will naturally be most interesting to a highly developed age
like our own, to an age making the demand which we have described for an
intellectual deliverance by means of the complete intelligence of its
own situation? Evidently, the other ages similarly developed, and making
the same demand. And what past literature will naturally be most
interesting to such an age as our own? Evidently, the literatures which
have most successfully solved for \emph{their} ages the problem which
occupies ours: the literatures which in their day and for their own
nation have adequately comprehended, have adequately represented, the
spectacle before them. A significant, a highly-developed, a culminating
epoch, on the one hand,---a comprehensive, a commensurate, an adequate
literature, on the other,---these will naturally be the objects of
deepest interest to our modern age. Such an epoch and such a literature
are, in fact, \emph{modern}, in the same sense in which our own age and
literature are modern; they are founded upon a rich past and upon an
instructive fulness of experience.

It may, however, happen that a great epoch is without a perfectly
adequate literature; it may happen that a great age, a great nation, has
attained a remarkable fulness of political and social development,
without intellectually taking the complete measure of itself, without
adequately representing that development in its literature. In this
case, the \emph{epoch}, the \emph{nation} itself, will still be an
object of the greatest interest to us; but the \emph{literature} will be
an object of less interest to us: the facts, the material spectacle, are
there; but the contemporary view of the facts, the intellectual
interpretation, are inferior and inadequate.

It may happen, on the other hand, that great authors, that a powerful
literature, are found in an age and nation less great and powerful than
themselves; it may happen that a literature, that a man of genius, may
arise adequate to the representation of a greater, a more highly
developed age than that in which they appear; it may happen that a
literature completely interprets its epoch, and yet has something over;
that it has a force, a richness, a geniality, a power of view which the
materials at its disposition are insufficient adequately to employ. In
such a case, the literature will be more interesting to us than the
epoch. The interpreting power, the illuminating and revealing intellect,
are there; but the spectacle on which they throw their light is not
fully worthy of them.

And I shall not, I hope, be thought to magnify too much my office if I
add, that it is to the poetical literature of an age that we must, in
general, look for the most perfect, the most adequate interpretation of
that age,---for the performance of a work which demands the most
energetic and harmonious activity of all the powers of the human mind.
Because that activity of the whole mind, that genius, as Johnson nobly
describes it, ``without which judgement is cold and knowledge is inert;
that energy which collects, combines, amplifies, and animates,'' is in
poetry at its highest stretch and in its most energetic exertion.

What we seek, therefore, what will most enlighten us, most contribute to
our intellectual deliverance, is the union of two things; it is the
coexistence, the simultaneous appearance, of a great epoch and a great
literature.

Now the culminating age in the life of ancient Greece I call, beyond
question, a great epoch; the life of Athens in the fifth century before
our era I call one of the highly developed, one of the marking, one of
the modern periods in the life of the whole human race. It has been
saidthat the ``Athens of Pericles was a vigorous man, at the summit of
his bodily strength and mental energy.'' There was the utmost energy of
life there, public and private; the most entire freedom, the most
unprejudiced and intelligent observation of human affairs. Let us
rapidly examine some of the characteristics which distinguish modern
epochs; let us see how far the culminating century of ancient Greece
exhibits them; let us compare it, in respect of them, with a much later,
a celebrated century; let us compare it with the age of Elizabeth in our
own country.

To begin with what is exterior. One of the most characteristic outward
features of a \emph{modern} age, of an age of advanced civilization, is
the banishment of the ensigns of war and bloodshed from the intercourse
of civil life. Crime still exists, and wars are still carried on; but
within the limits of civil life a circle has been formed within which
man can move securely, and develop the arts of peace uninterruptedly.
The private man does not go forth to his daily occupation prepared to
assail the life of his neighbour or to have to defend his own. With the
disappearance of the constant means of offence the occasions of offence
diminish; society at last acquires repose, confidence, and free
activity. An important inward characteristic, again, is the growth of a
tolerant spirit; that spirit which is the offspring of an enlarged
knowledge; a spirit patient of the diversities of habits and opinions.
Other characteristics are the multiplication of the conveniences of
life, the formation of taste, the capacity for refined pursuits. And
this leads us to the supreme characteristic of all: the intellectual
maturity of man himself; the tendency to observe facts with a critical
spirit; to search for their law, not to wander among them at random; to
judge by the rule of reason, not by the impulse of prejudice or caprice.

Well, now, with respect to the presence of all these characteristics in
the age of Pericles, we possess the explicit testimony of an immortal
work,---of the history of Thucydides. ``The Athenians first,'' he
says---speaking of the gradual development of Grecian society up to the
period when the Peloponnesian war commenced---``the Athenians first left
off the habit of wearing arms:'' that is, this mark of superior
civilization had, in the age of Pericles, become general in Greece, had
long been visible at Athens. In the time of Elizabeth, on the other
hand, the wearing of arms was universal in England and throughout
Europe. Again, the conveniences, the ornaments, the luxuries of life,
had become common at Athens at the time of which we are speaking. But
there had been an advance even beyond this; there had been an advance to
that perfection, that propriety of taste which prescribes the excess of
ornament, the extravagance of luxury. The Athenians had given up,
Thucydides says, had given up, although not very long before, an
extravagance of dress and an excess of personal ornament which, in the
first flush of newly-discovered luxury, had been adopted by some of the
richer classes.The height of civilization in this respect seems to have
been attained; there was general elegance and refinement of life, and
there was simplicity. What was the case in this respect in the
Elizabethan age? The scholar Casaubon, who settled in England in the
reign of James I., bears evidence to the want here, even at that time,
of conveniences of life which were already to be met with on the
continent of Europe. On the other hand, the taste for fantastic, for
excessive personal adornment, to which the portraits of the time bear
testimony, is admirably set forth in the work of a great novelist, who
was also a very truthful antiquarian---in the ``Kenilworth'' of Sir
Walter Scott. We all remember the description, in the thirteenth and
fourteenth chapters of the second volume of ``Kenilworth,'' of the
barbarous magnificence, the ``fierce vanities,'' of the dress of the
period.

Pericles praises the Athenians that they had discovered sources of
recreation for the spirit to counterbalance the labours of the body:
compare these, compare the pleasures which charmed the whole body of the
Athenian people through the yearly round of their festivals with the
popular shows and pastimes in ``Kenilworth.'' ``We have freedom,'' says
Pericles, ``for individual diversities of opinion and character; we do
not take offence at the tastes and habits of our neighbour if they
differ from our own.'' Yes, in Greece, in the Athens of Pericles, there
is toleration; but in England, in the England of the sixteenth
century?---the Puritans are then in full growth. So that with regard to
these characteristics of civilization of a modern spirit which we have
hitherto enumerated, the superiority, it will be admitted, rests with
the age of Pericles.

Let us pass to what we said was the supreme characteristic of a highly
developed, a modern age---the manifestation of a critical spirit, the
endeavour after a rational arrangement and appreciation of facts. Let us
consider one or two of the passages in the masterly introduction which
Thucydides, the contemporary of Pericles, has prefixed to his history.
What was hismotive in choosing the Peloponnesian War for his subject?
Because it was, in his opinion, the most important, the most instructive
event which had, up to that time, happened in the history of mankind.
What is his effort in the first twenty-three chapters of his history? To
place in their correct point of view all the facts which had brought
Grecian society to the point at which that dominant event found it; to
strip these facts of their exaggeration, to examine them critically. The
enterprises undertaken in the early times of Greece were on a much
smaller scale than had been commonly supposed. The Greek chiefs were
induced to combine in the expedition against Troy, not by their respect
for an oath taken by them all when suitors to Helen, but by their
respect for the preponderating influence of Agamemnon; the siege of Troy
had been protracted not so much by the valour of the besieged as by the
inadequate mode of warfare necessitated by the want of funds of the
besiegers. No doubt Thucydides' criticism of the Trojan war is not
perfect; but observe how in these and many other points he labours to
correct popular errors, to assign their true character to facts,
complaining, as he does so, of men's habit of \emph{uncritical}
reception of current stories. ``So little a matter of care to most
men,'' he says, ``is the search after truth, and so inclined are they to
take up any story which is ready to their hand.'' ``He himself,'' he
continues, ``has endeavoured to give a true picture, and believes that
in the main he has done so. For some readers his history may want the
charm of the uncritical, half-fabulous narratives of earlier writers;
but for such as desire to gain a clear knowledge of the past, and
thereby of the future also, which will surely, after the course of human
things, represent again hereafter, if not the very image, yet the near
resemblance of the past---if such shall judge my work to be profitable,
I shall be well content.''

What language shall we properly call this? It is \emph{modern} language;
it is the language of a thoughtful philosophic man of our own days; it
is the language of Burke or Niebuhr assigning the true aim of history.
And yet Thucydides is no mere literary man; no isolated thinker,
speaking far over the heads of his hearers to a future age---no: he was
a man of action, a man of the world, a man of his time. He represents,
at its best indeed, but he represents, the general intelligence of his
age and nation; of a nation the meanest citizens of which could follow
with comprehension the profoundly thoughtful speeches of Pericles.

Let us now turn for a contrast to a historian of the Elizabethan age,
also a man of great mark and ability, also a man of action, also a man
of the world, Sir Walter Ralegh. Sir Walter Ralegh writes the ``History
of the World,'' as Thucydides has written the ``History of the
Peloponnesian War''; let us hear his language; let us mark his point of
view; let us see what problems occur to him for solution. ``Seeing,'' he
says, ``that we digress in all the ways of our lives---yea, seeing the
life of man is nothing else but digression---I may be the better excused
in writing their lives and actions.'' What are the preliminary facts
which he discusses, as Thucydides discusses the Trojan War and the early
naval power of Crete, and which are to lead up to his main inquiry? Open
the table of contents of his first volume. You will find:---``Of the
firmament, and of the waters above the firmament, and whether there be
any crystalline Heaven, or any primum mobile.'' You will then
find:---``Of Fate, and that the stars have great influence, and that
their operations may diversely be prevented or furthered.'' Then you
come to two entire chapters on the place of Paradise, and on the two
chief trees in the garden of Paradise. And in what style, with what
power of criticism, does Ralegh treat the subjects so selected? I turn
to the 7th section of the third chapter of his first book, which treats
``Of their opinion which make Paradise as high as the moon, and of
others which make it higher than the middle region of the air.'' Thus he
begins the discussion of this opinion:---``Whereas Beda saith, and as
the schoolmen affirm Paradise to be a place altogether removed from the
knowledge of men (`locus a cognitione hominum remotissimus'),and
Barcephas conceived that Paradise was far in the east, but mounted above
the ocean and all the earth, and near the orb of the moon (which
opinion, though the schoolmen charge Beda withal, yet Pererius lays it
off from Beda and his master Rabanus); and whereas Rupertus in his
geography of Paradise doth not much differ from the rest, but finds it
seated next or nearest Heaven---'' So he states the error, and now for
his own criticism of it. ``First, such a place cannot be commodious to
live in, for being so near the moon it had been too near the sun and
other heavenly bodies. Secondly, it must have been too joint a neighbour
to the element of fire. Thirdly, the air in that region is so violently
moved and carried about with such swiftness as nothing in that place can
consist or have abiding. Fourthly,''---but what has been quoted is
surely enough, and there is no use in continuing.

Which is the ancient here, and which is the modern? Which uses the
language of an intelligentman of our own days? which a language wholly
obsolete and unfamiliar to us? Which has the rational appreciation and
control of his facts? which wanders among them helplessly and without a
clue? Is it our own countryman, or is it the Greek? And the language of
Ralegh affords a fair sample of the critical power, of the point of
view, possessed by the majority of intelligent men of his day; as the
language of Thucydides affords us a fair sample of the critical power of
the majority of intelligent men in the age of Pericles.

Well, then, in the age of Pericles we have, in spite of its antiquity, a
highly-developed, a modern, a deeply interesting epoch. Next comes the
question: Is this epoch adequately interpreted by its highest
literature? Now, the peculiar characteristic of the highest
literature---the poetry---of the fifth century in Greece before the
Christian era, is its \emph{adequacy}; the peculiar characteristic of
the poetry of Sophocles is its consummate, its unrivalled
\emph{adequacy}; that it represents the highly developed human nature of
that age---human nature developed in a number of directions,
politically, socially, religiously, morally developed---in its
completest and most harmonious development in all these directions;
while there is shed over this poetry the charm of that noble serenity
which always accompanies true insight. If in the body of Athenians of
that time there was, as we have said, the utmost energy of mature
manhood, public and private; the most entire freedom, the most
unprejudiced and intelligent observation of human affairs---in Sophocles
there is the same energy, the same maturity, the same freedom, the same
intelligent observation; but all these idealized and glorified by the
grace and light shed over them from the noblest poetical feeling. And
therefore I have ventured to say of Sophocles, that he ``saw life
steadily, and saw it whole.'' Well may we understand how Pericles---how
the great statesman whose aim was, it has been said, ``to realize in
Athens the idea which he had conceived of human greatness,'' and who
partly succeeded in his aim---should have been drawn to the great poet
whose works are the noblest reflection of his success.

I assert, therefore, though the detailed proof of the assertion must be
reserved for other opportunities, that, if the fifth century in Greece
before our era is a significant and modern epoch, the poetry of that
epoch---the poetry of Pindar, Æschylus, and Sophocles---is an adequate
representation and interpretation of it.

The poetry of Aristophanes is an adequate representation of it also.
True, this poetry regards humanity from the comic side; but there is a
comic side from which to regard humanity as well as a tragic one; and
the distinction of Aristophanes is to have regarded it from the true
point of view on the comic side. He too, like Sophocles, regards the
human nature of his time in its fullest development; the boldest
creations of a riotous imagination are in Aristophanes, as has been
justly said, based always upon the foundation of a serious thought:
politics, education, social life, literature---all the great modes in
which the human life of his day manifested itself---are the subjects of
his thoughts, and of his penetrating comment. There is shed, therefore,
over his poetry the charm, the vital freshness, which is felt when man
and his relations are from any side adequately, and therefore genially,
regarded. Here is the true difference between Aristophanes and Menander.
There has been preserved an epitome of a comparison by Plutarch between
Aristophanes and Menander, in which the grossness of the former, the
exquisite truth to life and felicity of observation of the latter, are
strongly insisted upon; and the preference of the refined, the learned,
the intelligent men of a later period for Menander loudly proclaimed.
``What should take a man of refinement to the theatre,'' asks Plutarch,
``except to see one of Menander's Plays? When do you see the theatre
filled with cultivated persons, except when Menander is acted? and he is
the favourite refreshment,'' he continues, ``to the overstrained mind of
the laborious philosopher.'' And every one knows thefamous line of
tribute to this poet by an enthusiastic admirer in antiquity:---``O Life
and Menander, which of you painted the other?'' We remember, too, how a
great English statesman is said to have declared that there was no lost
work of antiquity which he so ardently desired to recover as a play of
Menander. Yet Menander has perished, and Aristophanes has survived. And
to what is this to be attributed? To the instinct of self-preservation
in humanity. The human race has the strongest, the most invincible
tendency to \emph{live}, to \emph{develop} itself. It retains,it clings
to what fosters its life, what favours its development, to the
literature which exhibits it in its vigour; it rejects, it abandons what
does not foster its development, the literature which exhibits it
arrested and decayed. Now, between the times of Sophocles and Menander a
great check had befallen the development of Greece;---the failure of the
Athenian expedition to Syracuse, and the consequent termination of the
Peloponnesian War in a result unfavourable to Athens. The free expansion
of her growth was checked; one of the noblest channels of Athenian life,
that of political activity, had begun to narrow and to dry up. That was
the true catastrophe of the ancient world; it was then that the oracles
of the ancient world should have become silent, and that its gods should
have forsaken their temples; for from that date the intellectual and
spiritual life of Greece was left without an adequate material basis of
political and practical life; and both began inevitably to decay. The
opportunity of the ancient world was then lost, never to return; for
neither the Macedonian nor the Roman world, which possessed an adequate
material basis, possessed, like the Athens of earlier times, an adequate
intellect and soul to inform and inspire them; and there was left of the
ancient world, when Christianity arrived, of Greece only a head without
a body, and of Rome only a body without a soul.

It is Athens after this check, after this diminution of vitality,---it
is man with part of his life shorn away, refined and intelligent indeed,
but sceptical, frivolous, and dissolute,---which the poetry of Menander
represented. The cultivated, the accomplished might applaud the
dexterity, the perfection of the representation---might prefer it to the
free genial delineation of a more living time with which they were no
longer in sympathy. But the instinct of humanity taught it, that in the
one poetry there was the seed of life, in the other poetry the seed of
death; and it has rescued Aristophanes, while it has left Menander to
his fate.

In the flowering period of the life of Greece, therefore, we have a
culminating age, one of the flowering periods of the life of the human
race: in the poetry of that age we have a literature commensurate with
its epoch. It is most perfectly commensurate in the poetry of Pindar,
Æschylus, Sophocles, Aristophanes; these, therefore, will be the
supremely interesting objects in this literature; but the stages in
literature which led up to this point of perfection, the stages in
literature which led downward from it, will be deeply interesting also.
A distinguished person,\footnote{Mr.~Gladstone} who has lately been
occupying himself with Homer, has remarked that an undue preference is
given, in the studies of Oxford, to these poets over Homer. The
justification of such a preference, even if we put aside all
philological considerations, lies, perhaps, in what I have said. Homer
himself is eternally interesting; he is a greater poetical power than
even Sophocles or Æschylus; but his age is less interesting than
himself. Æschylus and Sophocles represent an age as interesting as
themselves; the names, indeed, in their dramas are the names of the old
heroic world, from which they were far separated; but these names are
taken, because the use of them permits to the poet that free and ideal
treatment of his characters which the highest tragedy demands; and into
these figures of the old world is poured all the fulness of life and of
thought which the new world had accumulated. This new world in its
maturity of reason resembles our own; and the advantage over Homer in
their greater significance for \emph{us}, which Æschylus and Sophocles
gain by belonging to this new world, morethan compensates for their
poetical inferiority to him.

Let us now pass to the Roman world. There is no necessity to accumulate
proofs that the culminating period of Roman history is to be classed
among the leading, the significant, the modern periods of the world.
There is universally current, I think, a pretty correct appreciation of
the high development of the Rome of Cicero and Augustus; no one doubts
that material civilization and the refinements of life were largely
diffused in it; no one doubts that cultivation of mind and intelligence
were widely diffused in it. Therefore, I will not occupy time by showing
that Cicero corresponded with his friends in the style of the most
accomplished, the most easy letter-writers of modern times; that Cæsar
did not write history like Sir Walter Ralegh. The great period of Rome
is, perhaps, on the whole, the greatest, the fullest, the most
significant period on record; it is certainly a greater, a fuller period
than the age of Pericles. It is an infinitely larger school for the men
reared in it; the relations of life are immeasurably multiplied, the
events which happen are on an immeasurably grander scale.The facts, the
spectacle of this Roman world, then, are immense: let us see how far the
literature, the interpretation of the facts, has been adequate.

Let us begin with a great poet, a great philosopher, Lucretius. In the
case of Thucydides I called attention to the fact that his habit of
mind, his mode of dealing with questions, were modern; that they were
those of an enlightened, reflecting man among ourselves. Let me call
attention to the exhibition in Lucretius of a modern \emph{feeling} not
less remarkable than the modern \emph{thought} in Thucydides. The
predominance of thought, of reflection, in modern epochs is not without
its penalties; in the unsound, in the over-tasked, in the
over-sensitive, it has produced the most painful, the most lamentable
results; it has produced a state of feeling unknown to less enlightened
but perhaps healthier epochs---the feeling of depression, the feeling of
\emph{ennui}. Depression and \emph{ennui}; these are the characteristics
stamped on how many of the representative works of modern times! they
are also the characteristics stamped on the poem of Lucretius. One of
the most powerful, the most solemn passages of the work of Lucretius,
one of the most powerful, the most solemn passages in the literature of
the whole world, is the well-known conclusion of the third book. With
masterly touches he exhibits the lassitude, the incurable tedium which
pursue men in their amusements; with indignant irony he upbraids them
for the cowardice with which they cling to a life which for most is
miserable; to a life which contains, for the most fortunate, nothing but
the old dull round of the same unsatisfying objects for ever presented.
``A man rushes abroad,'' he says, ``because he is sick of being at home;
and suddenly comes home again because he finds himself no whit easier
abroad. He posts as fast as his horses can take him to his country-seat:
when he has got there he hesitates what to do; or he throws himself down
moodily to sleep, and seeks forgetfulness in that; or he makes the best
of his way back to town again with the same speed as he fled from it.
Thus every one flies from himself.'' What a picture of \emph{ennui}! of
the disease of the most modern societies, the most advanced
civilizations! ``O man,'' he exclaims again, ``the lights of the world,
Scipio, Homer, Epicurus, are dead; wilt thou hesitate and fret at dying,
whose life is well-nigh dead whilst thou art yet alive; who consumest in
sleep the greater part of thy span, and when awake dronest and ceasest
not to dream; and carriest about a mind troubled with baseless fear, and
canst not find what it is that aileth thee when thou staggerest like a
drunken wretch in the press of thy cares, and welterest hither and
thither in the unsteady wandering of thy spirit!'' And again: ``I have
nothing more than you have already seen,'' he makes Nature say to man,
``to invent for your amusement; \emph{eadem sunt omnia semper}---all
things continue the same for ever.''

Yes, Lucretius is modern; but is he adequate? And how can a man
adequately interpret the activity of his age when he is not in sympathy
with it? Think of the varied, the abundant, the wide spectacle of the
Roman life of his day; think of its fulness of occupation, its energy of
effort. From these Lucretius withdraws himself, and bids his disciples
to withdraw themselves; he bids them to leave the business of the world,
and to apply themselves ``\emph{naturam cognoscere rerum}---to learn the
nature of things;'' but there is no peace, no cheerfulness for him
either in the world from which he comes, or in the solitude to which he
goes. With stern effort, with gloomy despair, he seems to rivet his eyes
on the elementary reality, the naked framework of the world, because the
world in its fulness and movement is too exciting a spectacle for his
discomposed brain. He seems to feel the spectacle of it at once
terrifying and alluring; and to deliver himself from it he has to keep
perpetually repeating his formula of disenchantment and annihilation. In
reading him, you understand the tradition which represents him as having
been driven mad by a poison administered as a love-charm by his
mistress, and as having composed his great work in the intervals of his
madness. Lucretius is, therefore, overstrained, gloom-weighted, morbid;
and he who is morbid is no adequate interpreter of his age.

I pass to Virgil; to the poetical name which of all poetical names has
perhaps had the most prodigious fortune; the name which for Dante, for
the Middle Age, represented the perfection of classical antiquity. The
perfection of classical antiquity Virgil does not represent; but far be
it from me to add my voice to those which have decried his genius;
nothing that I shall say is,or can ever be, inconsistent with a
profound, an almost affectionate veneration for him. But with respect to
him, as with respect to Lucretius, I shall freely ask the question,
\emph{Is he adequate?} Does he represent the epoch in which he lived,
the mighty Roman world of his time, as the great poets of the great
epoch of Greek life represented theirs, in all its fullness, in all its
significance?

From the very form itself of his great poem, the Æneid, one would be led
to augur that this was impossible. The epic form, as a form for
representing contemporary or nearly contemporary events, has attained,
in the poems of Homer, an unmatched, an immortal success; the epic form
as employed by learned poets for the reproduction of the events of a
past age has attained a very considerable success. But for \emph{this}
purpose, for the poetic treatment of the events of a \emph{past} age,
the epic form is a less vital form than the dramatic form.The great
poets of the modern period of Greece are accordingly, as we have seen,
the \emph{dramatic} poets. The chief of these---Æschylus, Sophocles,
Euripides, Aristophanes---have survived: the distinguished epic poets of
the same period---Panyasis, Chœrilus, Antimachus---though praised by the
Alexandrian critics, have perished in a common destruction with the
undistinguished. And what is the reason of this? It is, that the
dramatic form exhibits, above all, \emph{the actions of man as strictly
determined by his thoughts and feelings;} it exhibits, therefore, what
may be always accessible, always intelligible, always interesting. But
the epic form takes a wider range; it represents not only the thought
and passion of man, that which is universal and eternal, but also the
forms of outward life, the fashion of manners, the aspects of nature,
that which is local or transient. To exhibit adequately what is local
and transient, only a witness, a contemporary, can suffice. In the
\emph{reconstruction}, by learning and antiquarian ingenuity, of the
local and transient features of a past age, in their representation by
one who is not a witness or contemporary, it is impossible to feel the
liveliest kind of interest. What, for instance, is the most interesting
portion of the Æneid,---the portion where Virgil seems to be moving most
freely, and therefore to be most animated, most forcible? Precisely that
portion which has most a \emph{dramatic} character; the episode of Dido;
that portion where locality and manners are nothing---where persons and
characters are everything. We might presume beforehand, therefore, that
if Virgil, at a time when contemporary epic poetry was no longer
possible, had been inspired to represent human life in its fullest
significance, he would not have selected the epic form. Accordingly,
what is, in fact, the character of the poem, the frame of mind of the
poet? Has the poem the depth, the completeness of the poems of Æschylus
or Sophocles, of those adequate and consummate representations of human
life? Has the poet the serious cheerfulness of Sophocles, of a man who
has mastered the problem of human life, who knows its gravity, and is
therefore serious, but who knows that he comprehends it, and is
therefore cheerful? Over the whole of the great poem of Virgil, over the
whole Æneid, there rests an ineffable melancholy: not a rigid, a moody
gloom, like the melancholy of Lucretius; no, a sweet, a touching
sadness, but still a sadness; a melancholy which is at once a source of
charm in the poem, and a testimony to its incompleteness. Virgil, as
Niebuhr has well said, expressed no affected self-disparagement, but the
haunting, the irresistible self-dissatisfaction of his heart, when he
desired on his deathbed that his poem might be destroyed. A man of the
most delicate genius, the most rich learning, but of weak health, of the
most sensitive nature, in a great and overwhelming world; conscious, at
heart, of his inadequacy for the thorough spiritual mastery of that
world and its interpretation in a work of art; conscious of this
inadequacy---the one inadequacy, the one weak place in the mighty Roman
nature! This suffering, this graceful-minded, this finely-gifted man is
the most beautiful, the most attractive figure in literary history; but
he is not the adequate interpreter of the great period of Rome.

We come to Horace: and if Lucretius, if Virgil want cheerfulness, Horace
wants seriousness. Igo back to what I said of Menander: as with Menander
so it is with Horace: the men of taste, the men of cultivation, the men
of the world are enchanted with him; he has not a prejudice, not an
illusion, not a blunder. True! yet the best men in the best ages have
never been thoroughly satisfied with Horace. If human life were complete
without faith, without enthusiasm, without energy, Horace, like
Menander, would be the perfect interpreter of human life: but it is not;
to the best, to the most living sense of humanity, it is not; and
because it is not, Horace is inadequate. Pedants are tiresome, men of
reflection and enthusiasm are unhappy and morbid; therefore Horace is a
sceptical man of the world. Men of action are without ideas, men of the
world are frivolous and sceptical; therefore Lucretius is plunged in
gloom and in stern sorrow. So hard, nay, so impossible for most men is
it to develop themselves in their entireness; to rejoice in the variety,
the movement of human life with the children of the world; to be serious
over the depth, the significance of human life with the wise! Horace
warms himself before the transient fire of human animation and human
pleasure while he can, and is only serious when he reflects that the
fire must soon go out:---


\begin{verse} 

``Damna tamen celeres reparant cœlestia lunæ:\\Nos, ubi decidimus---``


\end{verse} 

`For nature there is renovation, but for man there is none!'---it is
exquisite, but it is notinterpretative and fortifying.

In the Roman world, then, we have found a highly modern, a deeply
significant, an interesting period---a period more significant and more
interesting, because fuller, than the great period of Greece; but we
have not a commensurate literature. In Greece we have seen a highly
modern, a most significant and interesting period, although on a scale
of less magnitude and importance than the great period of Rome; but
then, co-existing with the great epoch of Greece there is what is
wanting to that of Rome, a commensurate, an interesting literature.

The intellectual history of our race cannot be clearly understood
without applying to other ages, nations, and literatures the same method
of inquiry which we have been here imperfectly applying to what is
called classical antiquity. But enough has at least been said, perhaps,
to establish the absolute, the enduring interest of Greek literature,
and, above all, of Greek poetry.

%\newpage