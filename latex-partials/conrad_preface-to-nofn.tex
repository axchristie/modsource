\title{Preface to \emph{The Nigger of the ``Narcissus''}}
\author{}
\date{1914}

% This line adds an entry for each work into the table of contents.
%\addcontentsline{toc}{chapter}{Preface to \emph{The Nigger of the ``Narcissus''},  (1914)}
\addcontentsline{toc}{chapter}{Preface to \emph{The Nigger of the ``Narcissus''} (1914) \newline Joseph Conrad }

\renewcommand{\chaptername}{Conrad, Preface to \emph{The Nigger of the ``Narcissus''}}

\thispagestyle{plain}

%\chapter[Preface to \emph{The Nigger of the ``Narcissus''} (1914) \newline Joseph Conrad]{Preface to \emph{The Nigger of the ``Narcissus''}}

% BEGIN KLUDGY TITLE BIT %%%%%%%%%%%%%%%%%%%%%%%%%%%%%%%%%%%%%%%%%%%%%%%%%%%%%%%
% We're not using \maketitle; instead this is Chris's own 
% kludgey way of outputting the title. Note the 
% uncomfortable amount of finagling with \linespread, \noindent
% and \vspace to make it look okay.
\begin{raggedright}
{\Large \linespread{1.0} \noindent \textbf{Preface to \emph{The Nigger of the ``Narcissus''}} \par} 

{\large Joseph Conrad \par} 

\vspace{0.5em}
\end{raggedright}

\begin{raggedleft}
{\large \linespread{1.2} (1914) \par}
\end{raggedleft}
\vspace{1em}
% END KLUDGY TITLE BIT %%%%%%%%%%%%%%%%%%%%%%%%%%%%%%%%%%%%%%%%%%%%%%%%%%%%%%%%%

%\maketitle

%\section{ Preface to \emph{The Nigger of the ``Narcissus''}    }






A work that aspires, however humbly, to the condition of art should
carry its justification in every line. And art itself may be defined as
a single-minded attempt to render the highest kind of justice to the
visible universe, by bringing to light the truth, manifold and one,
underlying its every aspect. It is an attempt to find in its forms, in
its colours, in its light, in its shadows, in the aspects of matter and
in the facts of life what of each is fundamental, what is enduring and
essential---their one illuminating and convincing quality---the very
truth of their existence. The artist, then, like the thinker or the
scientist, seeks the truth and makes his appeal. Impressed by the aspect
of the world the thinker plunges into ideas, the scientist into
facts---whence, presently, emerging they make their appeal to those
qualities of our being that fit us best for the hazardous enterprise of
living. They speak authoritatively to our common-sense, to our
intelligence, to our desire of peace or to our desire of unrest; not
seldom to our prejudices, sometimes to our fears, often to our
egoism---but always to our credulity. And their words are heard with
reverence, for their concern is with weighty matters: with the
cultivation of our minds and the proper care of our bodies, with the
attainment of our ambitions, with the perfection of the means and the
glorification of our precious aims.

It is otherwise with the artist.

Confronted by the same enigmatical spectacle the artist descends within
himself, and in that lonely region of stress and strife, if he be
deserving and fortunate, he finds the terms of his appeal. His appeal is
made to our less obvious capacities: to that part of our nature which,
because of the warlike conditions of existence, is necessarily kept out
of sight within the more resisting and hard qualities---like the
vulnerable body within a steel armour. His appeal is less loud, more
profound, less distinct, more stirring---and sooner forgotten. Yet its
effect endures forever. The changing wisdom of successive generations
discards ideas, questions facts, demolishes theories. But the artist
appeals to that part of our being which is not dependent on wisdom: to
that in us which is a gift and not an acquisition---and, therefore, more
permanently enduring. He speaks to our capacity for delight and wonder,
to the sense of mystery surrounding our lives, to our sense of pity, and
beauty, and pain; to the latent feeling of fellowship with all
creation---and to the subtle but invincible conviction of solidarity
that knits together the loneliness of innumerable hearts, to the
solidarity in dreams, in joy, in sorrow, in aspirations, in illusions,
in hope, in fear, which binds men to each other, which binds together
all humanity---the dead to the living and the living to the unborn.

It is only some such train of thought, or rather of feeling, that can in
a measure explain the aim of the attempt, made in the tale which
follows, to present an unrestful episode in the obscure lives of a few
individuals out of all the disregarded multitude of the bewildered, the
simple and the voiceless. For, if any part of truth dwells in the belief
confessed above, it becomes evident that there is not a place of
splendour or a dark corner of the earth that does not deserve, if only a
passing glance of wonder and pity. The motive then, may be held to
justify the matter of the work; but this preface, which is simply an
avowal of endeavour, cannot end here---for the avowal is not yet
complete.

Fiction---if it at all aspires to be art---appeals to temperament. And
in truth it must be, like painting, like music, like all art, the appeal
of one temperament to all the other innumerable temperaments whose
subtle and resistless power endows passing events with their true
meaning, and creates the moral, the emotional atmosphere of the place
and time. Such an appeal to be effective must be an impression conveyed
through the senses; and, in fact, it cannot be made in any other way,
because temperament, whether individual or collective, is not amenable
to persuasion. All art, therefore, appeals primarily to the senses, and
the artistic aim when expressing itself in written words must also make
its appeal through the senses, if its high desire is to reach the secret
spring of responsive emotions. It must strenuously aspire to the
plasticity of sculpture, to the colour of painting, and to the magic
suggestiveness of music---which is the art of arts. And it is only
through complete, unswerving devotion to the perfect blending of form
and substance; it is only through an unremitting never-discouraged care
for the shape and ring of sentences that an approach can be made to
plasticity, to colour, and that the light of magic suggestiveness may be
brought to play for an evanescent instant over the commonplace surface
of words: of the old, old words, worn thin, defaced by ages of careless
usage.

The sincere endeavour to accomplish that creative task, to go as far on
that road as his strength will carry him, to go undeterred by faltering,
weariness or reproach, is the only valid justification for the worker in
prose. And if his conscience is clear, his answer to those who in the
fulness of a wisdom which looks for immediate profit, demand
specifically to be edified, consoled, amused; who demand to be promptly
improved, or encouraged, or frightened, or shocked, or charmed, must run
thus:---My task which I am trying to achieve is, by the power of the
written word to make you hear, to make you feel---it is, before all, to
make you \emph{see}. That---and no more, and it is everything. If I
succeed, you shall find there according to your deserts: encouragement,
consolation, fear, charm---all you demand---and, perhaps, also that
glimpse of truth for which you have forgotten to ask.

To snatch in a moment of courage, from the remorseless rush of time, a
passing phase of life, is only the beginning of the task. The task
approached in tenderness and faith is to hold up unquestioningly,
without choice and without fear, the rescued fragment before all eyes in
the light of a sincere mood. It is to show its vibration, its colour,
its form; and through its movement, its form, and its colour, reveal the
substance of its truth---disclose its inspiring secret: the stress and
passion within the core of each convincing moment. In a single-minded
attempt of that kind, if one be deserving and fortunate, one may
perchance attain to such clearness of sincerity that at last the
presented vision of regret or pity, of terror or mirth, shall awaken in
the hearts of the beholders that feeling of unavoidable solidarity; of
the solidarity in mysterious origin, in toil, in joy, in hope, in
uncertain fate, which binds men to each other and all mankind to the
visible world.

It is evident that he who, rightly or wrongly, holds by the convictions
expressed above cannot be faithful to any one of the temporary formulas
of his craft. The enduring part of them---the truth which each only
imperfectly veils---should abide with him as the most precious of his
possessions, but they all: Realism, Romanticism, Naturalism, even the
unofficial sentimentalism (which like the poor, is exceedingly difficult
to get rid of), all these gods must, after a short period of fellowship,
abandon him---even on the very threshold of the temple---to the
stammerings of his conscience and to the outspoken consciousness of the
difficulties of his work. In that uneasy solitude the supreme cry of Art
for Art itself, loses the exciting ring of its apparent immorality. It
sounds far off. It has ceased to be a cry, and is heard only as a
whisper, often incomprehensible, but at times and faintly encouraging.

Sometimes, stretched at ease in the shade of a roadside tree, we watch
the motions of a labourer in a distant field, and after a time, begin to
wonder languidly as to what the fellow may be at. We watch the movements
of his body, the waving of his arms, we see him bend down, stand up,
hesitate, begin again. It may add to the charm of an idle hour to be
told the purpose of his exertions. If we know he is trying to lift a
stone, to dig a ditch, to uproot a stump, we look with a more real
interest at his efforts; we are disposed to condone the jar of his
agitation upon the restfulness of the landscape; and even, if in a
brotherly frame of mind, we may bring ourselves to forgive his failure.
We understood his object, and, after all, the fellow has tried, and
perhaps he had not the strength---and perhaps he had not the knowledge.
We forgive, go on our way---and forget.

And so it is with the workman of art. Art is long and life is short, and
success is very far off. And thus, doubtful of strength to travel so
far, we talk a little about the aim---the aim of art, which, like life
itself, is inspiring, difficult---obscured by mists. It is not in the
clear logic of a triumphant conclusion; it is not in the unveiling of
one of those heartless secrets which are called the Laws of Nature. It
is not less great, but only more difficult.

To arrest, for the space of a breath, the hands busy about the work of
the earth, and compel men entranced by the sight of distant goals to
glance for a moment at the surrounding vision of form and color, of
sunshine and shadows; to make them pause for a look, for a sigh, for a
smile---such is the aim, difficult and evanescent, and reserved only for
a few to achieve. But sometimes, by the deserving and the fortunate,
even that task is accomplished. And when it is
accomplished---behold!---all the truth of life is there: a moment of
vision, a sigh, a smile---and the return to an eternal rest.

%\newpage