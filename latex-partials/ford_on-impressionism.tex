\title{On Impressionism}
\author{}
\date{1914}

% This line adds an entry for each work into the table of contents.
%\addcontentsline{toc}{chapter}{On Impressionism,  (1914)}
\addcontentsline{toc}{chapter}{On Impressionism (1914) \newline Ford Madox Ford {[}Hueffer{]} }

\renewcommand{\chaptername}{Ford {[}Hueffer{]}, On Impressionism}

\thispagestyle{plain}

%\chapter[On Impressionism (1914) \newline Ford Madox Ford {[}Hueffer{]}]{On Impressionism}

% BEGIN KLUDGY TITLE BIT %%%%%%%%%%%%%%%%%%%%%%%%%%%%%%%%%%%%%%%%%%%%%%%%%%%%%%%
% We're not using \maketitle; instead this is Chris's own 
% kludgey way of outputting the title. Note the 
% uncomfortable amount of finagling with \linespread, \noindent
% and \vspace to make it look okay.
\begin{raggedright}
{\Large \linespread{1.0} \noindent \textbf{On Impressionism} \par} 

{\large Ford Madox Ford {[}Hueffer{]} \par} 

\vspace{0.5em}
\end{raggedright}

\begin{raggedleft}
{\large \linespread{1.2} (1914) \par}
\end{raggedleft}
\vspace{1em}
% END KLUDGY TITLE BIT %%%%%%%%%%%%%%%%%%%%%%%%%%%%%%%%%%%%%%%%%%%%%%%%%%%%%%%%%

%\maketitle

%\section{ On Impressionism    }






\subsection{First Article}\label{first-article}

\subsubsection{I.}\label{i.}

These are merely some notes towards a working guide to Impressionism as
a literary method.

I do not know why I should have been especially asked to write about
Impressionism; even as far as literary Impressionism goes I claim no
Papacy in the matter. A few years ago, if anybody had called me an
Impressionist I should languidly have denied that I was anything of the
sort or that I knew anything about the school, if there could be said to
be any school. But one person and another in the last ten years has
called me Impressionist with such persistence that I have given up
resistance. I don't know; I just write books, and if someone attaches a
label to me I do not much mind.

I am not claiming any great importance for my work; I daresay it is all
right. At any rate, I am a perfectly self-conscious writer; I know
exactly how I get my effects, as far as those effects go. Then, if I am
in truth an Impressionist, it must follow that a conscientious and exact
account of how I myself work will be an account, from the inside, of how
Impressionism is reached, produced, or gets its effects. I can do no
more.

This is called egotism; but, to tell the truth, I do not see how
Impressionism can be anything else. Probably this school differs from
other schools, principally, in that it recognises, frankly, that all art
must be the expression of an ego, and that if Impressionism is to do
anything, it must, as the phrase is, go the whole hog. The difference
between the description of a grass by the agricultural correspondent of
the \emph{Times} newspaper and the description of the same grass by Mr
W.\textsubscript{H.}Hudson is just the difference---the measure of the
difference between the egos of the two gentlemen. The difference between
the description of any given book by a sound English reviewer and the
description of the same book by some foreigner attempting Impressionist
criticism is again merely a matter of the difference in the ego.

Mind, I am not saying that the non-Impressionist productions may not
have their values---their very great values. The Impressionist gives you
his own views, expecting you to draw deductions, since presumably you
know the sort of chap he is. The agricultural correspondent of the
\emph{Times}, on the other hand---and a jolly good writer he
is---attempts to give you, not so much his own impressions of a new
grass as the factual observations of himself and of as many as possible
other sound authorities. He will tell you how many blades of the new
grass will grow upon an acre, what height they will attain, what will be
a reasonable tonnage to expect when green, when sun-dried in the form of
hay or as ensilage. He will tell you the fattening value of the new
fodder in its various forms and the nitrogenous value of the manure
dropped by the so-fattened beasts. He will provide you, in short, with
reading that is quite interesting to the layman, since all facts are
interesting to men of good will; and the agriculturist he will provide
with information of real value. Mr.\textasciitilde{}Hudson, on the other
hand, will give you nothing but the pleasure of coming in contact with
his temperament, and I doubt whether, if you read with the greatest care
his description of false sea-buckthorn (*hippophae rhamnoides) you would
very willingly recognise that greenish-grey plant, with the spines and
the berries like reddish amber, if you came across it.

Or again---so at least I was informed by an editor the other day---the
business of a sound English reviewer is to make the readers of the paper
understand exactly what sort of a book it is that the reviewer is
writing about. Said the editor in question: ``You have no idea how many
readers your paper will lose if you employ one of those brilliant chaps
who write readable articles about books. You will get yourself deluged
with letter after letter from subscribers saying they have bought a book
on the strength of articles in your paper; that the book isn't in the
least what they expected, and that therefore they withdraw their
subscriptions.'' What the sound English reviewer, therefore, has to do
is to identify himself with the point of view of as large a number of
readers of the journal for which he may be reviewing, as he can easily
do, and then to give them as many facts about the book under
consideration as his allotted space will hold. To do this he must
sacrifice his personality, and the greater part of his readability. But
he will probably very much help his editor, since the great majority of
readers do not want to read anything that any reasonable person would
want to read; and they do not want to come into contact with the
personality of the critic, since they have obviously never been
introduced to him.

The ideal critic, on the other hand---as opposed to the so-exemplary
reviewer---is a person who can so handle words that from the first three
phrases any intelligent person---any foreigner, that is to say, and any
one of three inhabitants of these islands---any intelligent person will
know at once the sort of chap that he is dealing with. Letters of
introduction will therefore be unnecessary, and the intelligent reader
will know pretty well what sort of book the fellow is writing about
because he will know the sort of fellow the fellow is, I don't mean to
say that he would necessarily trust his purse, his wife, or his mistress
to the Impressionist critic's care. But that is not absolutely
necessary. The ambition, however, of my friend the editor was to let his
journal give the impression of being written by those who could be
trusted with the wives and purses---not, of course, the mistresses, for
there would be none---of his readers.

You will, perhaps, be beginning to see now what I am aiming at---the
fact that Impressionism is a frank expression of personality; the fact
that non-Impressionism is an attempt to gather together the opinions of
as many reputable persons as may be and to render them truthfully and
without exaggeration. (The Impressionist must always exaggerate.)

\subsubsection{II.}\label{ii.}

Let us approach this matter historically---as far as I know anything
about the history of Impressionism, though I must warn you that I am a
shockingly ill-read man. Here, then, are some examples: do you know, for
instance, Hogarth's drawing of the watchman with the pike over his
shoulder and the dog at his heels going in at a door, the whole being
executed in four lines ? Here it is:

\begin{figure}[htbp]
\centering
\includegraphics{images/ford_on-impressionism_01.png}
\end{figure}

Now, that is the high-watermark of Impressionism; since, if you look at
those lines for long enough, you will begin to see the watchman with his
slouch hat, the handle of the pike coming well down into the
cobble-stones, the knee-breeches, the leathern garters strapped round
his stocking, and the surly expression of the dog, which is bull-hound
with a touch of mastiff in it.

You may ask why, if Hogarth saw all these things, did he not put them
down on paper, and all that I can answer is that he made this drawing
for a bet. Moreover why, if you can see all these things for yourself,
should Hogarth bother to put them down on paper? You might as well
contend that Our Lord ought to have delivered a lecture on the state of
primary education in the Palestine of the year 32 or thereabouts,
together with the statistics of rickets and other infantile diseases
caused by neglect and improper feeding---a disquisition in the manner of
Mrs Sidney Webb. He preferred, however, to say: ``It were better that a
millstone were put about his neck and he were cast into the deep sea.''
The statement is probably quite incorrect; the statutory punishment
either here or in the next world has probably nothing to do with
millstones and so on, but Our Lord was, you see, an Impressionist, and
knew His job pretty efficiently. It is probable that He did not have
access to as many Blue Books or white papers as the leaders of the
Fabian Society, but, from His published utterances, one gathers that He
had given a good deal of thought to the subject of children.

I am not in the least joking---and God forbid that I should be thought
irreverent because I write like this. The point that I really wish to
make is, once again, that---that the Impressionist gives you, as a rule,
the fruits of his own observations and the fruits of his own
observations alone. He should be in this as severe and as solitary as
any monk. It is what he is in the world for. It is, for instance, not so
much his business to quote as to state his impressions---that the Holy
Scriptures are a good book, or a rotten book, or contain passages of
good reading interspersed with dulness; or suggest gems in a cavern, the
perfumes of aromatic woods burning in censers, or the rush of the feet
of camels crossing the deep sands, or the shrill sounds of long trumpets
borne by archangels---clear sounds of brass like those in that funny
passage in ``Aida.''

The passage in prose, however, which I always take as a working
model---and in writing this article I am doing no more than showing you
the broken tools and bits of oily rag which form my brains, since once
again I must disclaim writing with any authority on Impressionism---this
passage in prose occurs in a story by de Maupassant called \emph{La
Reine Hortense}. I spent, I suppose, a great part of ten years in
grubbing up facts about Henry VIII. I worried about his parentage, his
diseases, the size of his shoes, the price he gave for kitchen
implements, his relation to his wives, his knowledge of music, his
proficiency with the bow. I amassed, in short, a great deal of
information about Henry VIII. I wanted to write a long book about him,
but Mr.\textasciitilde{}Pollard, of the British Museum, got the
commission and wrote the book probably much more soundly. I then wrote
three long novels all about that Defender of the Faith. But I really
know---so delusive are reported facts---nothing whatever. Not one single
thing! Should I have found him affable, or terrifying, or seductive, or
royal, or courageous? There are so many contradictory facts; there are
so many reported interviews, each contradicting the other, so that
really all that I \emph{know} about this king could be reported in the
words of Maupassant, which, as I say, I always consider as a working
model. Maupassant is introducing one of his characters, who is possibly
gross, commercial, overbearing, insolent; who eats, possibly, too much
greasy food; who wears commonplace clothes---a gentleman about whom you
might write volumes if you wanted to give the facts of his existence.
But all that de Maupassant finds it necessary to say is: ``C'était un
monsieur à favoris rouges qui entrait toujours le premier.''

And that is all that I \emph{know} about Henry VIII.---that he was a
gentleman with red whiskers who always went first through a door.

\subsubsection{III.}\label{iii.}

Let us now see how these things work out in practice. I have a certain
number of maxims, gained mostly in conversation with Mr Conrad, which
form my working stock-in-trade. I stick to them pretty generally;
sometimes I throw them out of the window and just write whatever comes.
But the effect is usually pretty much the same. I guess I must be fairly
well drilled by this time and function automatically, as the Americans
say. The first two of my maxims are these :

Always consider the impressions that you are making upon the mind of the
reader, and always consider that the first impression with which you
present him will be so strong that it will be all that you can ever do
to efface it, to alter it or even quite slightly to modify it.
Maupassant's gentleman with red whiskers, who always pushed in front of
people when it was a matter of going through a doorway, will remain, for
the mind of the reader, that man and no other. The impression is as hard
and as definite as a tin-tack. And I rather doubt whether, supposing
Maupassant represented him afterwards as kneeling on the ground to wipe
the tears away from a small child who had lost a penny down a drain---I
doubt whether such a definite statement of fact would ever efface the
first impression from the reader's mind. They would think that the
gentleman with the red whiskers was perpetrating that act of benevolence
with ulterior motives---to impress the bystanders, perhaps.

Maupassant, however, uses physical details more usually as a method of
introduction of his characters than I myself do. I am inclined myself,
when engaged in the seductive occupation, rather to strike the keynote
with a speech than with a description of personality, or even with an
action. And, for that purpose, I should set it down, as a rule, that the
first speech of a character you are introducing should always be a
generalisation---since generalisations are the really strong indications
of character. Putting the matter exaggeratedly, you might say that, if a
gentleman sitting opposite you in the train remarked to you: ``I see the
Tories have won Leith Boroughs,'' you would have practically no guide to
that gentleman's character. But, if he said: ``Them bloody Unionists
have crept into Leith because the Labourites, damn them, have taken away
1,100 votes from us,'' you would know that the gentleman belonged to a
certain political party, had a certain social status, a certain degree
of education and a certain amount of impatience.

It is possible that such disquisitions on Impressionism in prose fiction
may seem out of place in a journal styled POETRY AND DRAMA. But I do not
think they are. For Impressionism, differing from other schools of art,
is founded so entirely on observation of the psychology of the
patron---and the psychology of the patron remains constant. Let me, to
make things plainer, present you with a quotation. Sings Tennyson:


\begin{verse} 

``And bats went round in fragrant skies, And wheeled or lit the filmy
shapes That haunt the dusk, with ermine capes And woolly breasts and
beady eyes.''


\end{verse} 

Now that is no doubt very good natural history, but it is certainly not
Impressionism, since no one watching a bat at dusk could see the ermine,
the wool or the beadiness of the eyes. These things you might read about
in books, or observe in the museum or at the Zoological Gardens. Or you
might pick up a dead bat upon the road. But to import into the record of
observations of one moment the observations of a moment altogether
different is not Impressionism. For Impressionism is a thing altogether
momentary.

I do not wish to be misunderstood. It is perfectly possible that the
remembrance of a former observation may colour your impression of the
moment, so that if Tennyson had said:


\begin{verse} 

``And we remembered they have ermine capes,''


\end{verse} 

he would have remained within the canons of Impressionism. But that was
not his purpose, which, whatever it was, was no doubt praiseworthy in
the extreme, because his heart was pure. It is, however, perfectly
possible that a piece of Impressionism should give a sense of two, of
three, of as many as you will, places, persons, emotions, all going on
simultaneously in the emotions of the writer. It is, I mean, perfectly
possible for a sensitised person, be he poet or prose writer, to have
the sense, when he is in one room, that he is in another, or when he is
speaking to one person he may be so intensely haunted by the memory or
desire for another person that he may be absent-minded or distraught.
And there is nothing in the canons of Impressionism, as I know it, to
stop the attempt to render those superimposed emotions. Indeed, I
suppose that Impressionism exists to render those queer effects of real
life that are like so many views seen through bright glass---through
glass so bright that whilst you perceive through it a landscape or a
backyard, you are aware that, on its surface, it reflects a face of a
person behind you. For the whole of life is really like that; we are
almost always in one place with our minds somewhere quite other.

And it is, I think, only Impressionism that can render that peculiar
effect; I know, at any rate, of no other method. It has, this school, in
consequence, certain quite strong canons, certain quite rigid unities
that must be observed. The point is that any piece of Impressionism,
whether it be prose, or verse, or painting, or sculpture, is the record
of the impression of a moment; it is not a sort of rounded, annotated
record of a set of circumstances---it is the record of the recollection
in your mind of a set of circumstances that happened ten years ago---or
ten minutes. It might even be the impression of the moment---but it is
the impression, not the corrected chronicle. I can make what I mean most
clear by a concrete instance.

Thus an Impressionist in a novel, or in a poem, will never render a long
speech of one of his characters verbatim, because the mind of the reader
would at once lose some of the illusion of the good faith of the
narrator. The mind of the reader will say: ``Hullo, this fellow is
faking this. He cannot possibly remember such a long speech word for
word.'' The Impressionist, therefore, will only record his impression of
a long speech. If you will try to remember what remains in your mind of
long speeches you heard yesterday, this afternoon or five years ago, you
will see what I mean. If to-day, at lunch at your club, you heard an
irascible member making a long speech about the fish, what you remember
will not be his exact words. However much his proceedings will have
amused you, you will not remember his exact words. What you will
remember is that he said that the sole was not a sole, but a blank,
blank, blank plaice; that the cook ought to be shot, by God he ought to
be shot. The plaice had been out of the water two years, and it had been
caught in a drain: all that there was of Dieppe about this Sole
Dieppoise was something that you cannot remember. You will remember this
gentleman's starting eyes, his grunts between words, that he was fond of
saying ``damnable, damnable, damnable.'' You will also remember that the
man at the same table with you was talking about morals, and that your
boots were too tight, whilst you were trying, in your under mind, to
arrange a meeting with some lady\ldots{}..

So that, if you had to render that scene or those speeches for purposes
of fiction, you would not give a word for word re-invention of sustained
sentences from the gentleman who was dissatisfied; or if you were going
to invent that scene, you would not so invent those speeches and set
them down with all the panoply of inverted commas, notes of exclamation.
No, you would give an impression of the whole thing, of the snorts, of
the characteristic exclamation, of your friend's disquisition on morals,
a few phrases of which you would intersperse into the monologue of the
gentleman dissatisfied with his sole. And you would give a sense that
your feet were burning, and that the lady you wanted to meet had very
clear and candid eyes. You would give a little description of her
hair\ldots{}..

In that way you would attain to the sort of odd vibration that scenes in
real life really have; you would give your reader the impression that he
was witnessing something real, that he was passing through an
experience\ldots{}.. You will observe also that you will have produced
something that is very like a Futurist picture---not a Cubist picture,
but one of those canvases that show you in one corner a pair of stays,
in another a bit of the foyer of a music hall, in another a fragment of
early morning landscape, and in the middle a pair of eyes, the whole
bearing the title of ``A Night Out.'' And, indeed, those Futurists are
only trying to render on canvas what Impressionists \emph{tel que moi}
have been trying to render for many years. (You may remember Emma's love
scene at the cattle show in \emph{Madame Bovary}.)

Do not, I beg you, be led away by the English reviewer's cant phrase to
the effect that the Futurists are trying to be literary and the plastic
arts can never be literary. Les Jeunes of to-day are trying all sorts of
experiments, in all sorts of media. And they are perfectly right to be
trying them.

\subsection{Second Article}\label{second-article}

I have been trying to think what are the objections to Impressionism as
I understand it---or rather what alternative method could be found. It
seems to me that one is an Impressionist because one tries to produce an
illusion of reality---or rather the business of Impressionism is to
produce that illusion. The subject is one enormously complicated and is
full of negatives. Thus the Impressionist author is sedulous to avoid
letting his personality appear in the course of his book. On the other
hand, his whole book, his whole poem is merely an expression of his
personality. Let me illustrate exactly what I mean. You set out to write
a story, or you set out to write a poem, and immediately your attempt
becomes one creating an illusion, You attempt to involve the reader
amongst the personages of the story or in the atmosphere of the poem.
You do this by presentation and by presentation and again by
presentation. The moment you depart from presentation, the moment you
allow yourself, as a poet, to introduce the ejaculation:


\begin{verse} 

``O Muse Pindarian, aid me to my theme;''


\end{verse} 

or the moment that, as a story-teller, you permit yourself the luxury of
saying :

\begin{quote}
``Now, gentle reader, is my heroine not a very sweet and oppressed
lady?''---
\end{quote}

at that very moment your reader's illusion that he is present at an
affair in real life or that he has been transported by your poem into an
atmosphere entirely other than that of his arm-chair or his
chimney-corner---at that very moment that illusion will depart. Now the
point is this:

The other day I was discussing these matters with a young man whose
avowed intention is to sweep away Impressionism. And, after I had
energetically put before him the views that I have here expressed, he
simply remarked : ``Why try to produce an illusion?'' To which I could
only reply: ``Why then write?''

I have asked myself frequently since then why one should try to produce
an illusion of reality in the mind of one's reader. Is it just an
occupation like any other---like postage-stamp collecting, let us
say---or is it the sole end and aim of art? I have spent the greater
portion of my working life in preaching that particular doctrine: is it
possible, then, that I have been entirely wrong ?

Of course it is possible for any man to be entirely wrong; but I confess
myself to being as yet unconverted. The chief argument of my futurist
friend was that producing an illusion causes the writer so much trouble
as not to be worth while. That does not seem to me to be an argument
worth very much because---and again I must say it seems to me---the
business of an artist is surely to take trouble, but this is probably
doing my friend's position, if not his actual argument, an injustice. I
am aware that there are quite definite aesthetic objections to the
business of producing an illusion. In order to produce an illusion you
must justify; in order to justify you must introduce a certain amount of
matter that may not appear germane to your story or to your poem.
Sometimes, that is to say, it would appear as if for the purpose of
proper bringing out of a very slight Impressionist sketch the artist
would need an altogether disproportionately enormous frame; a frame
absolutely monstrous. Let me again illustrate exactly what I mean. It is
not sufficient to say: ``Mr Jones was a gentleman who had a strong
aversion to rabbit-pie.'' It is not sufficient, that is to say, if Mr
Jones's dislike for rabbit-pie is an integral part of your story. And it
is quite possible that a dislike for one form or other of food might
form the integral part of a story. Mr Jones might be a hard-worked
coal-miner with a well-meaning wife, whom he disliked because he was
developing a passion for a frivolous girl. And it might be quite
possible that one evening the well-meaning wife, not knowing her
husband's peculiarities, but desiring to give him a special and extra
treat, should purchase from a stall a couple of rabbits and spend many
hours in preparing for him a pie of great succulence, which should be a
solace to him when he returns, tired with his labours and rendered
nervous by his growing passion for the other lady. The rabbit-pie would
then become a symbol---a symbol of the whole tragedy of life. It would
symbolize for Mr Jones the whole of his wife's want of sympathy for him
and the whole of his distaste for her; his reception of it would
symbolize for Mrs Jones the whole hopelessness of her life, since she
had expended upon it inventiveness, sedulous care, sentiment, and a good
will. From that position, with the rabbit-pie always in the centre of
the discussion, you might work up to the murder of Mrs Jones, to Mr
Jones's elopement with the other lady---to any tragedy that you liked.
For indeed the position contains, as you will perceive, the whole
tragedy of life.

And the point is this, that if your tragedy is to be absolutely
convincing, it is not sufficient to introduce the fact of Mr Jones's
dislike for rabbit-pie by the bare statement. According to your
temperament you must sufficiently account for that dislike. You might do
it by giving Mr Jones a German grandmother, since all Germans have a
peculiar loathing for the rabbit and regard its flesh as unclean. You
might then find it necessary to account for the dislike the Germans have
for these little creatures; you might have to state that this dislike is
a self-preservative race instinct, since in Germany the rabbit is apt to
eat certain poisonous fungi, so that one out of every ten will cause the
death of its consumer, or you might proceed with your justification of
Mr Jones's dislike for rabbit-pie along different lines. You might say
that it was a nervous aversion caused by having been violently thrashed
when a boy by his father at a time when a rabbit-pie was upon the table.
You might then have to go on to justify the nervous temperament of Mr
Jones by saying that his mother drank or that his father was a man too
studious for his position. You might have to pursue almost endless
studies in the genealogy of Mr Jones; because, of course, you might want
to account for the studiousness of Mr Jones's father by making him the
bastard son of a clergyman, and then you might want to account for the
libidinous habits of the clergyman in question. That will be simply a
matter of your artistic conscience.

You have to make Mr Jones's dislike for rabbits convincing. You have to
make it in the first place convincing to your reader alone; but the odds
are that you will try to make it convincing also to yourself, since you
yourself in this solitary world of ours will be the only reader that you
really and truly know. Now all these attempts at justification, all
these details of parentage and the like, may very well prove
uninteresting to your reader. They are, however, necessary if your final
effect of murder is to be a convincing impression.

But again, if the final province of art is to convince, its first
province is to interest. So that, to the extent that your justification
is uninteresting, it is an artistic defect. It may sound paradoxical,
but the truth is that your Impressionist can only get his strongest
effects by using beforehand a great deal of what one may call
non-Impressionism. He will make, that is to say, an enormons impression
on his reader's mind by the use of three words. But very likely each one
of those three words will be prepared for by ten thousand other words.
Now are we to regard those other words as being entirely unnecessary, as
being, that is to say, so many artistic defects? That I take to be my
futurist friend's ultimate assertion,

Says he: ``All these elaborate conventions of Conrad or of Maupassant
give the reader the impression that a story is being told---all these
meetings of bankers and master-mariners in places like the Ship Inn at
Greenwich, and all Maupassant's dinner-parties, always in the politest
circles, where a countess or a fashionable doctor or someone relates a
passionate or a pathetic or a tragic or a merely grotesque incident---as
you have it, for instance, in the `Contes de la Bécasse'---all this
machinery for getting a story told is so much waste of time. A story is
a story; why not just tell it anyhow? You can never tell what sort of an
impression you will produce upon a reader. Then why bother about
Impressionism? Why not just chance your luck?''

There is a good deal to be said for this point of view. Writing up to my
own standards is such an intolerable labour and such a thankless job,
since it can't give me the one thing in the world that I desire---that
for my part I am determined to drop creative writing for good and all.
But I, like all writers of my generation, have been so handicapped that
there is small wonder that one should be tired out. On the one hand the
difficulty of getting hold of any critical guidance was, when I was a
boy, insuperable. There was nothing. Criticism was non-existent;
self-conscious art was decried; you were supposed to write by
inspiration; you were the young generation with the vine-leaves in your
hair, knocking furiously at the door. On the other hand, one writes for
money, for fame, to excite the passion of love, to make an impression
upon one's time. Well, God knows what one writes for. But it is certain
that one gains neither fame nor money; certainly one does not excite the
passion of love, and one's time continues to be singularly unimpressed.

But young writers to-day have a much better chance, on the æsthetic side
at least. Here and there, in nooks and corners, they can find someone to
discuss their work, not from the point of view of goodness or badness or
of niceness or of nastiness, but from the simple point of view of
expediency. The moment you can say: ``Is it expedient to print
\emph{vers libre} in long or short lines, or in the form of prose, or
not to print it at all, but to recite it?''---the moment you can find
someone to discuss these expediences calmly, or the moment that you can
find someone with whom to discuss the relative values of justifying your
character or of abandoning the attempt to produce an illusion of
reality---at that moment you are very considerably helped; whereas an
admirer of your work might fall down and kiss your feet and it would not
be of the very least use to you.

\subsubsection{II.}\label{ii.-1}

This adieu, like Herrick's, to poesy, may seem to be a digression.
Indeed it is; and indeed it isn't. It is, that is to say, a digression
in the sense that it is a statement not immediately germane to the
argument that I am carrying on. But it is none the less an insertion
fully in accord with the canons of Impressionism as I understand it. For
the first business of Impressionism is to produce an impression, and the
only way in literature to produce an impression is to awaken interest.
And, in a sustained argument, you can only keep interest awakened by
keeping alive, by whatever means you may have at your disposal, the
surprise of your reader. You must state your argument; you must
illustrate it, and then you must stick in something that appears to have
nothing whatever to do with either subject or illustration, so that the
reader will exclaim: ``What the devil is the fellow driving at?'' And
then you must go on in the same way---arguing, illustrating and
startling and arguing, startling and illustrating---until at the very
end your contentions will appear like a ravelled skein. And then, in the
last few lines, you will draw towards you the master-string of that
seeming confusion, and the whole pattern of the carpet, the whole design
of the net-work will be apparent.

This method, you will observe, founds itself upon analysis of the human
mind. For no human being likes listening to long and sustained
arguments. Such listening is an effort, and no artist has the right to
call for any effort from his audience. A picture should come out of its
frame and seize the spectator.

Let us now consider the audience to which the artist should address
himself. Theoretically a writer should be like the Protestant angel, a
messenger of peace and goodwill towards all men. But, inasmuch as the
Wingless Victory appears monstrously hideous to a Hottentot, and a
beauty of Tunis detestable to the inhabitants of these fortunate
islands, it is obvious that each artist must adopt a frame of mind, less
Catholic possibly, but certainly more Papist, and address himself, like
the angel of the Vulgate, only \emph{hominibus bonæ voluntatis}. He must
address himself to such men as be of goodwill; that is to say, he must
typify for himself a human soul in sympathy with his own; a silent
listener who will be attentive to him, and whose mind acts very much as
his acts. According to the measure of this artist's identity with his
species, so will be the measure of his temporal greatness. That is why a
book, to be really popular, must be either extremely good or extremely
bad. For Mr Hall Caine has millions of readers; but then Guy de
Maupassant and Flaubert have tens of millions.

I suppose the proposition might be put in another way. Since the great
majority of mankind are, on the surface, vulgar and trivial---the stuff
to fill graveyards---the great majority of mankind will be easily and
quickly affected by art which is vulgar and trivial. But, inasmuch as
this world is a very miserable purgatory for most of us sons of
men---who remain stuff with which graveyards are filled---inasmuch as
horror, despair and incessant strivings are the lot of the most trivial
of humanity, who endure them as a rule with commonsense and
cheerfulness---so, if a really great master strike the note of horror,
of despair, of striving, and so on, he will stir chords in the hearts of
a larger number of people than those who are moved by the merely vulgar
and the merely trivial. This is probably why Madame Bovary has sold more
copies than any book ever published, except, of course, books purely
religious. But the appeal of religious books is exactly similar.

It may be said that the appeal of Madame Bovary is largely sexual. So it
is, but it is only in countries like England and the United States that
the abominable tortures of sex---or, if you will, the abominable
interests of sex---are not supposed to take rank alongside of the
horrors of lost honour, commercial ruin, or death itself. For all these
things are the components of life, and each is of equal importance.

So, since Flaubert is read in Russia, in Germany, in France, in the
United States, amongst the non-Anglo-Saxon population, and by the
immense populations of South America, he may be said to have taken for
his audience the whole of the world that could possibly be expected to
listen to a man of his race. (I except, of course, the Anglo-Saxons who
cannot be confidently expected to listen to snything other than the
words produced by Mr George Edwardes, and musical comedy in general.)

My futurist friend again visited me yesterday, and we discussed this
very question of audiences. Here again he said that I was entirely
wrong. He said that an artist should not address himself to
\emph{l'homme moyen sensuel}, but to intellectuals, to people who live
at Hampstead and wear no hats. (He withdrew his contention later.)

I maintain on my own side that one should address oneself to the cabmen
round the corner, but this also is perhaps an exaggeration. My friend's
contention on behalf of the intellectuals was not so much due to his
respect for their intellects. He said that they knew the A B C of an
art, and that it is better to address yourself to an audience that knows
the A B C of an art than to an audience entirely untrammelled by such
knowledge. In this I think he was wrong, for the intellectuals are
persons of very conventional mind, and they acquire as a rule
simultaneously with the A B C of any art the knowledge of so many
conventions that it is almost impossible to make any impression upon
their minds. Hampstead and the hatless generally offer an impervious
front to futurisms, simply because they have imbibed from Whistler and
the Impressionists the convention that painting should not be literary.
Now every futurist picture tells a story; so that rules out futurism.
Similarly with the cubists. Hampstead has imbibed, from God knows where,
the dogma that all art should be based on life, or should at least draw
its inspiration and its strength from the representation of nature, So
there goes cubism, since cubism is non-representational, has nothing to
do with life, and has a quite proper contempt of nature.

When I produced my argument that one should address oneself to the
cabmen at the corner, my futurist friend at once flung to me the jeer
about Tolstoi and the peasant. Now the one sensible thing in the long
drivel of nonsense with which Tolstoi misled this dull world was the
remark that art should be addressed to the peasant. My futurist friend
said that that was sensible for an artist living in Russia or Roumania,
but it was an absurd remark to be let fall by a critic living on Campden
Hill. His view was that you cannot address yourself to the peasant
unless that peasant have evoked folk-song or folk-lores. I don't know
why that was his view, but that was his view.

It seems to me to be nonsensical, even if the inner meaning of his
dictum was that art should be addressed to a community of practising
artists. Art, in fact, should be addressed to those who are not
preoccupied. It is senseless to address a Sirventes to a man who is
going mad with love, and an Imagiste poem will produce little effect
upon another man who is going through the bankruptcy court.

It is probable that Tolstoi thought that in Russia the non-preoccupied
mind was to be found solely amongst the peasant class, and that is why
he said that works of art should be addressed to the peasant. I don't
know how it may be in Russia, but certainly in Occidental Europe the
non-preoccupied mind---which is the same thing as the peasant
intelligence---is to be found scattered throughout every grade of
society. When I used just now the instances of a man made for love, or
distracted by the prospect of personal ruin, I was purposely misleading.
For a man mad as a hatter for love of a worthless creature, or a man
maddened by the tortures of bankruptcy, by dishonour or by failure, may
yet have, by the sheer necessity of his nature, a mind more receptive
than most other minds. The mere craving for relief from his personal
thoughts may make him take quite unusual interest in a work of art. So
that is not preoccupation in my intended sense, but for a moment the
false statement crystallised quite clearly what I was aiming at.

The really impassible mind is not the mind quickened by passion, but the
mind rendered slothful by preoccupation purely trivial. The ``English
gentleman'' is, for instance, an absolutely hopeless being from this
point of view. His mind is so taken up by considerations of what is good
form, of what is good feeling, of what is even good fellowship; he is so
concerned to pass unnoticed in the crowd; he is so set upon having his
room like everyone else's room, that he will find it impossible to
listen to any plea for art which is exceptional, vivid, or startling.
The cabman, on the other hand, does not mind being thought a vulgar sort
of bloke; in consequence he will form a more possible sort of audience.
On the other hand, amongst the purely idler classes it is perfectly
possible to find individuals who are so firmly and titularly gentle folk
that they don't have to care a damn what they do. These again are
possible audiences for the artist. The point is really, I take it, that
the preoccupation that is fatal to art is the moral or the social
preoccupation. Actual preoccupations matter very little. Your cabman may
drive his taxi through exceedingly difficult streets; he may have
half-a-dozen close shaves in a quarter of an hour. But when those things
are over they are over, and he has not the necessity of a cabman. His
point of view as to what is art, good form, or, let us say, the proper
relation of the sexes, is unaffected. He may be a hungry man, a thirsty
man, or even a tired man, but he will not necessarily have his finger
upon his moral pulse, and he will not hold as æsthetic dogma the idea
that no painting must tell a story, or the moral dogma that passion only
becomes respectable when you have killed it.

It is these accursed dicta that render an audience hopeless to the
artist, that render art a useless pursuit and the artist himself a
despised individual.

So that those are the best individuals for an artist's audience who have
least listened to accepted ideas---who are acquainted with deaths at
street corners, with the marital infidelities of crowded courts, with
the goodness of heart of the criminal, with the meanness of the
undetected or the sinless, who know the queer odd jumble of negatives
that forms our miserable and hopeless life. If I had to choose as reader
I would rather have one who had never read anything before but the
Newgate Calendar, or the records of crime, starvation and divorce in the
Sunday paper---I would rather have him for a reader than the man who had
discovered the song that the sirens sang, or had by heart the whole of
the \emph{Times} Literary Supplement, from its inception to the present
day. Such a peasant intelligence will know that this is such a queer
world that anything may be possible. And that is the type of
intelligence that we need.

Of course, it is more difficult to find these intelligences in the town
than in the rural districts. A man thatching all day long has time for
many queer thoughts; so has a man who from sunrise to sunset is trimming
a hedge into shape with a bagging hook. I have, I suppose, myself
thought more queer thoughts when digging potatoes than at any other time
during my existence. It is, for instance, very queer if you are digging
potatoes in the late evening, when it has grown cool after a very hot
day, to thrust your hand into the earth after a potato and to find that
the earth is quite warm---is about flesh-heat. Of course, the clods
would be warm because the sun would have been shining on them all day,
and the air gives up its heat much quicker than the earth. But it is
none the less a queer sensation.

Now, if the person experiencing that sensation have what I call a
peasant intelligence, he will just say that it is a queer thing and will
store it away in his mind along with his other experiences. It will go
along with the remembrance of hard frost, of fantastic icycles, the
death of rabbits pursued by stoats, the singularly quick ripening of
corn in a certain year, the fact that such and such a man was overlooked
by a wise woman and so died because, his wife, being tired of him, had
paid the wise woman five sixpences which she had laid upon the table in
the form of a crown; or along with the other fact that a certain man
murdered his wife by the use of a packet of sheep dip which he had
stolen from a field where the farmer was employed at lamb washing. All
these remembrances he will have in his mind, not classified under any
headings of social reformers, or generalized so as to fulfil any fancied
moral law.

But the really dangerous person for the artist will be the gentleman
who, chancing to put his hand into the ground and to find it about as
warm as the breast of a woman, if you could thrust your hand between her
chest and her stays, will not accept the experience as an experience,
but will start talking about the breast of mother-nature. This last man
is the man whom the artist should avoid, since he will regard phenomena
not as phenomena, but as happenings, with which he may back up
preconceived dogmas---as, in fact, so many sticks with which to beat a
dog.

No, what the artist needs is the man with the quite virgin mind---the
man who will not insist that grass must always be painted green, because
all the poets, from Chaucer till the present day, had insisted on
talking about the green grass, or the green leaves, or the green straw.

Such a man, if he comes to your picture and sees you have painted a
haycock bright purple will say:

``Well, I have never myself observed a haycock to be purple, but I can
understand that if the sky is very blue and the sun is setting very red,
the shady side of the haycock might well appear to be purple.'' That is
the kind of peasant intelligence that the artist needs for his audience.

And the whole of Impressionism comes to this: having realized that the
audience to which you will address yourself must have this particular
peasant intelligence, or, if you prefer it, this particular and virgin
openness of mind, you will then figure to yourself an individual, a
silent listener, who shall be to yourself the \emph{homo bonæ
voluntatis}---man of goodwill. To him, then, you will address your
picture, your poem, your prose story, or your argument. You will seek to
capture his interest; you will seek to hold his interest. You will do
this by methods of surprise, of fatigue, by passages of sweetness in
your language, by passages suggesting the sudden and brutal shock of
suicide. You will give him passages of dulness, so that your bright
effects may seem more bright; you will alternate, you will dwell for a
long time upon an intimate point; you will seek to exasperate so that
you may the better enchant. You will, in short, employ all the devices
of the prostitute. If you are too proud for this you may be the better
gentleman or the better lady, but you will be the worse artist. For the
artist must always be humble and humble and again humble, since before
the greatness of his task he himself is nothing. He must again be
outrageous, since the greatness of his task calls for enormous excesses
by means of which he may recoup his energies. That is why the artist is,
quite rightly, regarded with suspicion by people who desire to live in
tranquil and ordered society.

But one point is very important. The artist can never write to satisfy
himself---to get, as the saying is, something off the chest. He must not
write propaganda which it is his desire to write; he must not write
rolling periods, the production of which gives him a soothing feeling in
his digestive organs or wherever it is. He must write always so as to
satisfy that other fellow---that other fellow who has too clear an
intelligence to let his attention be captured or his mind deceived by
special pleadings in favour of any given dogma. You must not write so as
to improve him, since he is a much better fellow than yourself, and you
must not write so as to influence him, since he is a granite rock, a
peasant intelligence, the gnarled bole of a sempiternal oak, against
which you will dash yourself in vain, It is in short no pleasant kind of
job to be a conscious artist. You won't have any vine-leaves in your
poor old hair; you won't just dash your quill into an inexhaustible
ink-well and pour out fine frenzies. No, you will be just the skilled
workman doing his job with drill or chisel or mallet. And you will get
precious little out of it. Only, just at times, when you come to look
again at some work of yours that you have quite forgotten, you will say,
``Why, that is rather well done.'' That is all.

%\newpage