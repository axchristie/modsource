\title{Conclusion to \emph{Studies in the History of the Renaissance}}
\author{}
\date{1873}

% This line adds an entry for each work into the table of contents.
%\addcontentsline{toc}{chapter}{Conclusion to \emph{Studies in the History of the Renaissance},  (1873)}
\addcontentsline{toc}{chapter}{Conclusion to \emph{Studies in the History of the Renaissance} (1873) \newline Walter H. Pater }

\renewcommand{\chaptername}{Pater, Conclusion to \emph{Studies in the History of the Renaissance}}

\thispagestyle{plain}

%\chapter[Conclusion to \emph{Studies in the History of the Renaissance} (1873) \newline Walter H. Pater]{Conclusion to \emph{Studies in the History of the Renaissance}}

% BEGIN KLUDGY TITLE BIT %%%%%%%%%%%%%%%%%%%%%%%%%%%%%%%%%%%%%%%%%%%%%%%%%%%%%%%
% We're not using \maketitle; instead this is Chris's own 
% kludgey way of outputting the title. Note the 
% uncomfortable amount of finagling with \linespread, \noindent
% and \vspace to make it look okay.
\begin{raggedright}
{\Large \linespread{1.0} \noindent \textbf{Conclusion to \emph{Studies in the History of the Renaissance}} \par} 

{\large Walter H. Pater \par} 

\vspace{0.5em}
\end{raggedright}

\begin{raggedleft}
{\large \linespread{1.2} (1873) \par}
\end{raggedleft}
\vspace{1em}
% END KLUDGY TITLE BIT %%%%%%%%%%%%%%%%%%%%%%%%%%%%%%%%%%%%%%%%%%%%%%%%%%%%%%%%%

%\maketitle

%\section{ Conclusion to \emph{Studies in the History of the Renaissance}    }






\begin{quote}
\emph{Λέγει που Ἡράκλειτοσ ὅτι πἀυτα xωρεῖ καὶ ούδέυ μέυει.}
\end{quote}

To regard all things and principles of things as inconstant modes or
fashions has more and more become the tendency of modern thought. Let us
begin with that which is without---our physical life. Fix upon it in one
of its more exquisite intervals, the moment, for instance, of delicious
recoil from the flood of water in summer heat. What is the whole
physical life in that moment but a combination of natural elements to
which science gives their names? But these elements, phosphorus and lime
and delicate fibres, are present not in the human body alone: we detect
them in places most remote from it. Our physical life is a perpetual
motion of them---the passage of the blood, the wasting and repairing of
the lenses of the eye, the modification of the tissues of the brain by
every ray of light and sound---processes which science reduces to
simpler and more elementary forces. Like the elements of which we are
composed, the action of these forces extends beyond us; it rusts iron
and ripens corn. Far out on every side of us those elements are
broadcast, driven by many forces; and birth and gesture and death and
the springing of violets from the grave are but a few out of ten
thousand resulting combinations. That clear perpetual outline of face
and limb is but an image of ours under which we group them---a design in
a web, the actual threads of which pass out beyond it. This at least of
flame-like our life has, that it is but the concurrence, renewed from
moment to moment, of forces parting sooner or later on their ways.

Or if we begin with the inward world of thought and feeling, the
whirlpool is still more rapid, the flame more eager and devouring. There
it is no longer the gradual darkening of the eye and fading of colour
from the wall,---the movement of the shore side, where the water flows
down indeed, though in apparent rest,---but the race of the midstream, a
drift of momentary acts of sight and passion and thought. At first sight
experience seems to bury us under a flood of external objects, pressing
upon us with a sharp importunate reality, calling us out of ourselves in
a thousand forms of action. But when reflection begins to act upon those
objects they are dissipated under its influence; the cohesive force is
suspended like a trick of magic; each object is loosed into a group of
impressions---colour, odour, texture---in the mind of the observer. And
if we continue to dwell on this world, not of objects in the solidity
with which language invests them, but of impressions unstable,
flickering, inconsistent, which burn and are extinguished with our
consciousness of them, it contracts still further; the whole scope of
observation is dwarfed to the narrow chamber of the individual mind.
Experience, already reduced to a swarm of impressions, is ringed round
for each one of us by that thick wall of personality through which no
real voice has ever pierced on its way to us, or from us to that which
we can only conjecture to be without. Every one of those impressions is
the impression of the individual in his isolation, each mind keeping as
a solitary prisoner its own dream of a world.

Analysis goes a step farther still, and tells us that those impressions
of the individual to which, for each one of us, experience dwindles
down, are in perpetual flight; that each of them is limited by time, and
that as time is infinitely divisible, each of them is infinitely
divisible also; all that is actual in it being a single moment, gone
while we try to apprehend it, of which it may ever be more truly said
that it has ceased to be than that it is. To such a tremulous wisp
constantly reforming itself on the stream, to a single sharp impression,
with a sense in it, a relic more or less fleeting, of such moments gone
by, what is \emph{real} in our life fines itself down. It is with the
movement, the passage and dissolution of impressions, images,
sensations, that analysis leaves off,---that continual vanishing away,
that strange, perpetual weaving and unweaving of ourselves.

\emph{Philosophiren}, says Novalis, \emph{ist dephlegmatisiren
vivificiren}. The service of philosophy, of religion and culture as
well, to the human spirit, is to startle it into a sharp and eager
observation. Every moment some form grows perfect in hand or face; some
tone on the hills or the sea is choicer than the rest; some mood of
passion or insight or intellectual excitement is irresistibly real and
attractive for us,---for that moment only. Not the fruit of experience,
but experience itself is the end. A counted number of pulses only is
given to us of a variegated, dramatic life. How may we see in them all
that is to be seen in them by the finest senses? How can we pass most
swiftly from point to point, and be present always at the focus where
the greatest number of vital forces unite in their purest energy?

To burn always with this hard, gem-like flame, to maintain this ecstasy,
is success in life. Failure is to form habits: for habit is relative to
a stereotyped world; meantime it is only the roughness of the eye that
makes any two persons, things, situations, seem alike. While all melts
under our feet, we may well catch at any exquisite passion, or any
contribution to knowledge that seems, by a lifted horizon, to set the
spirit free for a moment, or any stirring of the senses, strange dyes,
strange flowers, and curious odours, or work of the artist's hands, or
the face of one's friend. Not to discriminate every moment some
passionate attitude in those about us, and in the brilliance of their
gifts some tragic dividing of forces on their ways is, on this short day
of frost and sun, to sleep before evening. With this sense of the
splendour of our experience and of its awful brevity, gathering all we
are into one desperate effort to see and touch, we shall hardly have
time to make theories about the things we see and touch. What we have to
do is to be for ever curiously testing new opinions and courting new
impressions, never acquiescing in a facile orthodoxy of Comte or of
Hegel, or of our own. Theories, religious or philosophical ideas, as
points of view, instruments of criticism, may help us to gather up what
might otherwise pass unregarded by us. \emph{La philosophie, c'est la
microscope de la pensée}. The theory, or idea, or system, which requires
of us the sacrifice of any part of this experience, in consideration of
some interest into which we cannot enter, or some abstract morality we
have not identified with ourselves, or what is only conventional, has no
real claim upon us.

One of the most beautiful places in the writings of Rousseau is that in
the sixth book of the `Confessions,' where he describes the awakening in
him of the literary sense. An undefinable taint of death had always
clung about him, and now in early manhood he believed himself stricken
by mortal disease. He asked himself how he might make as much as
possible of the interval that remained; and he was not biassed by
anything in his previous life when he decided that it must be by
intellectual excitement, which he found in the clear, fresh writings of
Voltaire. Well, we are all \emph{condamnés}, as Victor Hugo says:
\emph{les hommes sont tous condamnés a morte avec des sursis indéfinis}:
we have an interval, and then our place knows us no more. Some spend
this interval in listlessness, some in high passions, the wisest in art
and song. For our one chance is in expanding that interval, in getting
as many pulsations as possible into the given time. High passions give
one this quickened sense of life, ecstasy and sorrow of love, political
or religious enthusiasm, or the `enthusiasm of humanity.' Only, be sure
it is passion, that it does yield you this fruit of a quickened,
multiplied consciousness. Of this wisdom, the poetic passion, the desire
of beauty, the love of art for art's sake has most; for art comes to you
professing frankly to give nothing but the highest quality to your
moments as they pass, and simply for those moments' sake.

%\newpage